%%%%%%%%%%%%%%%%%%%%%%%%%%%%%%%%%%%%%%%%%%%%%%%%%%%%%%%%%%%%%%%%%%%%%%%%%%%
% This is a sample header for a sample dissertation. Fill in the name,
% and the other information. LaTeX will work out the table of
% content, the list of figures and of tables for you.
%%%%%%%%%%%%%%%%%%%%%%%%%%%%%%%%%%%%%%%%%%%%%%%%%%%%%%%%%%%%%%%%%%%%%%%%%%%

\newpage
\thispagestyle{empty}




% ******* Title page *******
% **************************

\begin{onehalfspacing}
\begin{center}

\centering


\title{Praca inżynierska}
\author{Autor: Imię Nazwisko}


{\fontsize{17}{17}\selectfont
\textsc{Uniwersytet Śląski \\[.3cm]
Wydział Informatyki i Nauki o Materiałach  \\[.3cm]
Informatyka Inżynierska  \\[2.5cm]}
\textbf{Grzegorz Galios \\[.3cm]}




\large 
{Zastosowanie algorytmów ewolucyjnych w procesie nauki sztucznych sieci neuronowych} \\[.5cm]
\textsc{Praca dyplomowa inżynierska}
\end{center} ~\\[3cm]
% Jeśli tytuł pracy zajmuje 2 linijki, wartość [2.3cm] zamieniamy na [3.1cm], jeśli tylko jedną - na [3.9cm] i odwrotnie - zwiększając liczbę linijek o jedną (do czterech) zmieniamy na [1.5cm] itd.

\large
\begin{flushright}
dr Rafał Skinderowicz \\
\end{flushright}

\begin{bottompar}
\begin{flushright}
Sosnowiec, 2020
\end{flushright}
\end{bottompar}
\end{onehalfspacing}

% \singlespacing
% \newpage
% \thispagestyle{empty}
% \mbox{}


% %ABSTRACT
% \begin{abstract}
% The abstract will go here.... \\
% W tym miejscu można umieścić abstrakt pracy. W przeciwnym wypadku należy usunąć/zakomentować ninijeszy fragment kodu.
% \end{abstract}
% %END OF ABSTRACT


% \doublespacing
% \newpage
% \thispagestyle{empty}
% \mbox{}

%\pagestyle{empty}
\pagenumbering{Roman}
\setcounter{page}{0} \pagestyle{plain}

\pagestyle{fancy}