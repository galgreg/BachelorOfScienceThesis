\documentclass[twoside,a4paper,12pt]{extreport} %,draft,openright

\usepackage{polski}
\usepackage[utf8]{inputenc} 
\usepackage{gensymb}
\usepackage{epsf}

\newif\ifpdf
\ifx\pdfoutput\undefined
\pdffalse
\else
\pdfoutput=1
\pdftrue
\fi
\ifpdf
\usepackage[pdftex]{graphicx}
\pdfcompresslevel=9
\else
\usepackage{graphicx}
\fi
\usepackage{mdwlist}
\usepackage{subfigure}
\usepackage{latexsym,amssymb}
\usepackage{setspace,cite}
\usepackage{indentfirst}
\usepackage{mathtools}
\usepackage{url}
\usepackage[justification=centering]{caption}
\usepackage{multirow}
\usepackage[figuresright]{rotating}
\usepackage{csquotes}
\usepackage{listings,xcolor}
\usepackage{float}
\usepackage[T1]{fontenc}
\usepackage{mathptmx}
\usepackage{layout}
\newfloat{lstfloat}{htbp}{lop}
\floatname{lstfloat}{Listing}
\usepackage{enumitem}
\def\lstfloatautorefname{Listing} % needed for hyperref/auroref

\newenvironment{bottompar}{\par\vspace*{\fill}}{\clearpage}
\BeforeBeginEnvironment{figure}{\vskip-2ex}
\AfterEndEnvironment{figure}{\vskip-1ex}
\lstset{showstringspaces=false}

% for margins left, right top bottom
\usepackage{anysize}
\marginsize{3cm}{2.5cm}{2.5cm}{2.5cm}
\let\origdoublepage\cleardoublepage    %%komenda wstawiająca czyste kartki
\newcommand{\clearemptydoublepage}{%
  \clearpage
  {\pagestyle{empty}\origdoublepage}%
}
\let\cleardoublepage\clearemptydoublepage


%\usepackage{draft} %draft option - doesn't put full figures in -
            % useful when editing
%does the headers on the pages - keep in
\usepackage{fancyhdr}

%omitting any of these makes the thesis compile without the omitted
%chapter - good for editing single chapters.
%\includeonly{header,appendix}

\usepackage{etoolbox}% http://ctan.org/pkg/etoolbox
\makeatletter
\patchcmd{\@makechapterhead}{\vspace*{50\p@}}{}{}{}% Removes space above \chapter head
\makeatother
\usepackage{titlesec}
\titlespacing\section{0pt}{4pt plus 1pt minus 2pt}{0pt plus 1pt minus 2pt}
\titlespacing\subsection{0pt}{4pt plus 0.5pt minus 2pt}{0pt plus 0.5pt minus 2pt}
\titlespacing\subsubsection{0pt}{4pt plus 0.5pt minus 2pt}{0pt plus 0pt minus 2pt}
\begin{document}
\DeclareGraphicsExtensions{.pdf,.png,.eps}
\newpage

%Puts page numbering of preamble in roman and of main body of thesis in
%arabic. Also defines how chapters and sections are made
\pagenumbering{arabic}
\setcounter{page}{1} \pagestyle{fancy}
\renewcommand{\chaptermark}[1]{\markboth{\chaptername%
\ \thechapter:\,\ #1}{}}
\renewcommand{\sectionmark}[1]{\markright{\thesection\,\ #1}}

%DEFINES TITLE PAGE, and contains abstract, acknowledgements, etc.

%%%%%%%%%%%%%%%%%%%%%%%%%%%%%%%%%%%%%%%%%%%%%%%%%%%%%%%%%%%%%%%%%%%%%%%%%%%
% This is a sample header for a sample dissertation. Fill in the name,
% and the other information. LaTeX will work out the table of
% content, the list of figures and of tables for you.
%%%%%%%%%%%%%%%%%%%%%%%%%%%%%%%%%%%%%%%%%%%%%%%%%%%%%%%%%%%%%%%%%%%%%%%%%%%

\newpage
\thispagestyle{empty}




% ******* Title page *******
% **************************

\begin{onehalfspacing}
\begin{center}

\centering


\title{Praca inżynierska}
\author{Autor: Imię Nazwisko}


{\fontsize{17}{17}\selectfont
\textsc{Uniwersytet Śląski \\[.3cm]
Wydział Informatyki i Nauki o Materiałach  \\[.3cm]
Informatyka Inżynierska  \\[2.5cm]}
\textbf{Grzegorz Galios \\[.3cm]}




\large 
{Zastosowanie algorytmów ewolucyjnych w procesie nauki sztucznych sieci neuronowych} \\[.5cm]
\textsc{Praca dyplomowa inżynierska}
\end{center} ~\\[3cm]
% Jeśli tytuł pracy zajmuje 2 linijki, wartość [2.3cm] zamieniamy na [3.1cm], jeśli tylko jedną - na [3.9cm] i odwrotnie - zwiększając liczbę linijek o jedną (do czterech) zmieniamy na [1.5cm] itd.

\large
\begin{flushright}
dr Rafał Skinderowicz \\
\end{flushright}

\begin{bottompar}
\begin{flushright}
Sosnowiec, 2020
\end{flushright}
\end{bottompar}
\end{onehalfspacing}

% \singlespacing
% \newpage
% \thispagestyle{empty}
% \mbox{}


% %ABSTRACT
% \begin{abstract}
% The abstract will go here.... \\
% W tym miejscu można umieścić abstrakt pracy. W przeciwnym wypadku należy usunąć/zakomentować ninijeszy fragment kodu.
% \end{abstract}
% %END OF ABSTRACT


% \doublespacing
% \newpage
% \thispagestyle{empty}
% \mbox{}

%\pagestyle{empty}
\pagenumbering{Roman}
\setcounter{page}{0} \pagestyle{plain}

\pagestyle{fancy}

\newpage
\textbf{Oświadczenie autora pracy}\\

Ja, niżej podpisany: \\

\hspace{1.5cm} imię (imiona) i nazwisko: Grzegorz Galios \\

\hspace{1.5cm} autor pracy dyplomowej pt. 
\textit{„Zastosowanie algorytmów ewolucyjnych w procesie nauki sztucznych sieci
neuronowych”}\\

Numer albumu: 315850 \\

Student Wydziału Informatyki i Nauki o Materiałach Uniwersytetu Śląskiego w Katowicach \\ 

kierunku studiów: Informatyka Inżynierska – studia stacjonarne I stopnia \\

specjalności: Inżynieria Systemów Informatycznych \\

Oświadczam, że w/w. praca dyplomowa:   
\begin{itemize}
\item została przygotowana przeze mnie samodzielnie $^1$,  
\item nie narusza praw autorskich w rozumieniu ustawy z dnia 4 lutego 1994 r. o prawie autorskim i prawach pokrewnych (tekst jednolity Dz. U. z 2006 r. Nr 90, poz. 631, z późn. zm.) oraz dóbr osobistych chronionych prawem cywilnym,  
\item nie zawiera danych i informacji, które uzyskałem w sposób niedozwolony,  
\item nie była podstawą nadania dyplomu uczelni wyższej lub tytułu zawodowego ani mnie, ani innej osobie.  
\end{itemize}
Oświadczam również, że treść pracy dyplomowej zamieszczonej przeze mnie w Archiwum Prac Dyplomowych jest identyczna z treścią zawartą w wydrukowanej wersji pracy.  Jestem świadomy odpowiedzialności karnej za złożenie fałszywego oświadczenia.\\\\\\

\begin{minipage}{0.4\textwidth}
\begin{flushleft}
\centering
\textbf{...............................}\\
\textbf{Data}
\end{flushleft}
\end{minipage}
\begin{minipage}{0.4\textwidth}
\begin{flushright}
\textbf{...............................}\\
\textbf{Podpis autora pracy}
\end{flushright}
\end{minipage}\\\\

$^1$. uwzględniając merytoryczny wkład promotora (w ramach prowadzonego seminarium dyplomowego)

\tableofcontents
\newpage

%sets up headers for lefthand and righthand pages. To alter, edit
%these lines and the chaptermark/sectionmark lines above
%\addtolength{\headheight}{3pt} \fancyhead{}
%\fancyhead[LE]{\sl\leftmark} \fancyhead[LO,RE]{\rm\thepage}
%\fancyhead[RO]{\sl\rightmark} \fancyfoot[C,L,E]{}
\pagenumbering{arabic}
\fancyhead[LE,RO]{\slshape \rightmark}
\fancyhead[LO,RE]{\slshape \leftmark}
\fancyfoot[C]{\thepage}

\setlength{\parskip}{1ex} %odstępy między akapitami
%\singlespacing
%\doublespacing
\onehalfspacing

\chapter{Wstęp} \label{rozdz.wstep} 
\section{Cel i zakres pracy}
\section{Teza pracy}
\section{Metoda badawcza}
\section{Ograniczenia tematu}

\chapter{Samochody autonomiczne}

\section{Opis problemu}
Autonomiczny samochód to pojazd zdolny do interpretacji swojego otoczenia i bezpiecznego poruszania się w nim bez potrzeby ludzkiej ingerencji \cite{synopsys:whatIsAutonomousCar}. Taki pojazd potrafi robić to samo, co doświadczony kierowca robiłby prowadząc swój samochód.

\subsection{Poziomy automatyzacji sterowania}
Organizacja SAE (Society of Automotive Engineers) definiuje skalę, zakładającą sześć poziomów automatyzacji sterowania pojazdów \cite{synopsys:sixLevelsOfCarAutonomy}:
\begin{enumerate*}
\item Poziom 0 - sterowanie w pełni ręczne (\textbf{No Driving Automation}). Dotyczy większości samochodów poruszających się obecnie po drogach.
\item Poziom 1 - wsparcie kierowcy (\textbf{Driver Assistance}). Najniższy poziom automatyzacji. Samochód posiada prosty system wspomagający pojedyncze aspekty sterowania, np. kontrola kierownicy lub kontrola przyspieszenia samochodu (tzw. dynamiczny tempomat). 
Dynamiczny tempomat kwalifikuje się na poziom 1, ponieważ kierowca musi samodzielnie kontrolować pozostałe aspekty sterowania samochodem.
\item Poziom 2 - częściowa automatyzacja sterowania (\textbf{Partial Driving Automation}). Oznacza stosowanie zaawansowanych systemów wspomagania kierowcy (ADAS - Advanced Driver Assistant Systems) \cite{adas:opis}. Samochód potrafi kontrolować zarówno kierownicę jak również prędkość pojazdu. Samochód nie jest jednak w pełni autonomiczny, więc osoba siedząca za kierownicą może w każdej chwili przejąć kontrolę nad samochodem. Przykładem systemu będącego na Poziomie 2 jest Tesla Autopilot.
\item Poziom 3 - warunkowa automatyzacja sterowania (\textbf{Conditional Driving Automation}). Samochody posiadają system detekcji otoczenia, więc mogą podejmować decyzje, takie jak przyspieszanie obok powoli poruszającego się pojazdu. Jednakże wciąż wymagają nadzoru człowieka. Kierowca musi być przygotowany, iż w każdej chwili będzie musiał przejąć kontrolę nad samochodem, jeśli system nie poradzi sobie w danej sytuacji. \\
Obecna generacja samochodu Audi A8 oferuje system Traffic Jam Pilot \cite{audi:trafficJamPilot}, który zapewnia autonomiczność sterowania na Poziomie 3. Jest to pierwszy przypadek, w którym samochód produkowany seryjnie oferuje taki system \cite{audi:newAudiA8ConditionalAutomated}.
\item Poziom 4 - wysoka automatyzacja sterowania (\textbf{High Driving Automation}). Samochód potrafi interweniować jeśli "coś pójdzie nie tak", np. jeśli dojdzie do błędu systemu. W tej sytuacji, samochód nie potrzebuje ludzkiej ingerencji \textit{w większości przypadków}. Jednakże, kierowca wciąż może przejąć kontrolę nad samochodem. Przykłady występowania samochodów o autonomiczności na Poziomie 4:
\begin{itemize*}
\item Samochody elektryczne marki Navya - Autonom Shuttle oraz Autonom Cab \cite{motorAuthority:navyaCars}
\item Firma Waymo udostępniła usługę taksówkarską w Arizonie, gdzie samochody z autonomicznością na Poziomie 4 przewożą klientów  \cite{waymoAutonomousTaxi}.
\end{itemize*}
\item Poziom 5 - pełna automatyzacja sterowania (Full Driving Automation). Zachowanie pojazdu nigdy nie wymaga uwagi człowieka. Samochody Poziomu 5 nie będą nawet posiadać kierownicy ani pedałów. Będą wolne od ograniczeń obszarowych (geofencing), na których mogą się poruszać. W pełni autonomiczne samochody są obecnie testowane w kilku miejscach na świecie, jednakże żaden z tych systemów nie jest dostępny publicznie.
\end{enumerate*}

\begin{figure}[h]
\begin{center}
\includegraphics[width=15cm]{resources/figures/automation_levels.jpg}
\caption{Poziomy automatyzacji sterowania samochodem (według SAE)}
\label{CarAutomationLevels}
\end{center}
\end{figure}

Skala ta została zaakceptowana przez Amerykański Departament Transportu.

\subsection{Zasady działania}
Autonomiczne samochody korzystają z wszelkiego rodzaju urządzeń rejestrujących stan otoczenia, takich jak czujniki, radary i kamery. Radary monitorują pozycję obiektów znajdujących się wokół pojazdu, kamery wideo rozpoznają znaki drogowe oraz tor ruchu innych obiektów. Czujniki LIDAR (Light Detection and Ranging) odbijają impulsy świetlne od otoczenia samochodu. Dzięki temu są w stanie odmierzać dystans, rozpoznawać pobocza dróg oraz identyfikować oznaczenia pasa ruchu.
Czujniki ultradźwiękowe, montowane w kołach, rozpoznają krawężniki oraz inne pojazdy podczas parkowania.

Skomplikowane oprogramowanie przetwarza dane z urządzeń wejściowych, oblicza parametry dalszej jazdy (takie jak kierunek lub prędkość) i wysyła je do siłowników samochodu, które kontrolują przyspieszenie, hamowanie i kąt skrętu kierownicy.

\subsection{Wyzwania stawiane producentom}
W pełni autonomiczne samochody są poddawane wyczerpującym testom w kilku miejscach na świecie, jednakże żaden z nich nie jest dostępny publicznie. Od wprowadzenia takich systemów do produkcji seryjnej dzieli nas jeszcze wiele lat. Ponadto, przed projektantami i producentami stoi szereg wyzwań na polu technologicznym, prawnym, środowiskowym, a nawet filozoficznym. \\
Część z tych wyzwań to:
\begin{enumerate*}
\item LIDAR - problem wzajemnego zakłócania sygnałów w sytuacji gdy wiele autonomicznych samochodów będzie jechało blisko siebie. W sytuacji stosowania wielu różnych częstotliwości, czy zakres częstotliwości będzie wystarczająco szeroki aby obsłużyć masową produkcję takich pojazdów?
\item Warunki pogodowe - co się dzieje, gdy samochody autonomiczne podróżują w bardzo ciężkich warunkach? Jeśli śnieg zalega na drodze, oznaczenia pasa ruchu są niewidoczne. Jak kamery i czujniki będą rozpoznawać pasy ruchu, jeśli oznaczenia są zamazywane przez wodę, lód, olej lub błoto?
\item Regulacje prawne - wiele kwestii wymaga uregulowania, co nie będzie łatwe, bo dotyczy bardzo skomplikowanych zagadnień. Najistotniejszym z nich będzie kwestia \textbf{ustalenia odpowiedzialności prawnej za ewentualne wypadki powodowane przez samochody autonomiczne}.

\newpage
Cytując Norberta Biedrzyckiego z portalu Business Insider \cite{businessInsider:autonomiczneAutaPrawo}: \\
\begin{itshape}
,,Kto w sytuacji kolizji, zagrożenia czy nawet utraty życia, staje się przedmiotem sporu, możliwego pozwu, czy odszkodowania? Kto ma ponosić zasadniczą odpowiedzialność w sytuacji, w której giną osoby w wypadku z udziałem autonomicznych pojazdów? Czy kłopoty będzie miał producent algorytmu, w oparciu o który porusza się autonomiczne auto na drodze, firma produkująca takie auto, czy kierowca – właściciel? A jeśli ten ostatni spowoduje wypadek, czy może on liczyć na wystarczające rozwiązania związane z ubezpieczeniem, które zapewnią mu komfort użytkowania? \\

Na tym przykładzie widać wyraźnie, że możliwości technologiczne wyprzedzają nasze czasy. Tutaj właściwie wszystko jest możliwe – bo całe floty autonomicznych ciężarówek czekają już, by wyruszyć w drogę, a myśl o tym, by budować usługi taksówkarskie w oparciu o autonomiczne pojazdy, zaprząta głowę niejednemu biznesmenowi. \\

Okazuje się jednak, że podstawową barierą, która jeszcze długo będzie blokować zmiany na światowych drogach, będą kwestie niewystarczających uregulowań prawnych.''
\end{itshape}
\end{enumerate*}

\subsection{Korzyści z istnienia samochodów autonomicznych}
Scenariusze w których fakt wykorzystania pojazdów autonomicznych podnosi poziom wygody i jakości życia wydają się nieograniczone. Osoby starsze i niepełnosprawne mogą zyskać niezależność. Można wysłać psa do weterynarza. Albo przywieźć dzieciom na obóz brakujące rzeczy, których zapomniały zapakować.

Oprócz tego, z samochodami autonomicznymi wiąże się ogromną nadzieję, iż przyczynią się do drastycznego obniżenia emisji dwutlenku węgla do atmosfery. W artykule z 2017 roku, opublikowanego przez organizację ITDP, zamieszczono analizę na temat przyszłości transportu drogowego w kontekście wykorzystania samochodów autonomicznych \cite{itdp:urbanTransportRevolutions}. Autorzy analizy zwrócili uwagę na trzy tendencje we współczesnym spojrzeniu na transport, które jeśli będą występować jednocześnie, uwolnią pełny potencjał drzemiący w samochodach autonomicznych:
\begin{itemize*}
\item Automatyzacja
\item Elektryfikacja
\item Współdzielenie środka transportu (ridesharing)
\end{itemize*}
\newpage
Autorzy szacują, iż dzięki tym trzem ,,rewolucjom'', do roku 2050 będzie można osiągnąć:
\begin{enumerate*}
\item Zmniejszenie natężenia ruchu miejskiego o 30\%
\item Oszczędności w kosztach transportu na poziomie 40\%
\item Redukcję dwutlenku węgla emitowanego przez miasta na poziomie 80\% w skali światowej
\end{enumerate*}

\begin{figure}[h]
\begin{center}
\includegraphics[width=15cm]{resources/figures/itdp_infographic.jpg}
\caption{Przyszłość transportu miejskiego wg organizacji ITDP}
\label{IdtpTransportFuture}
\end{center}
\end{figure}

\section{Podejścia stosowane do rozwiązania problemu}
Stworzenie samochodu autonomicznego, który będzie w stanie samodzielnie poruszać się po otoczeniu i służyć jako środek transportu dla milionów ludzi na świecie, nie jest zadaniem trywialnym. Wymaga rozwiązania szeregu problemów, związanych z tym zagadnieniem. Takimi problemami mogą być chociażby rozpoznawanie znaków drogowych czy predykcja zachowań innych uczestników ruchu drogowego. Każdy z tych problemów może zostać rozpatrzony na wiele sposobów, istnieje zatem wiele alternatywnych rozwiązań dla danego problemu. Często nie można stwierdzić, czy dane rozwiązanie jest lepsze od reszty. Po prostu każde takie rozwiązanie ma swoją specyfikę, a co za tym idzie swoje wady i zalety.

Dlatego jest oczywiste, że firmy podejmujące się zbudowania samochodu autonomicznego mogą przyjmować różne, czasem skrajne podejścia w kwestii rozwiązań stosowanych w procesie tworzenia takiej maszyny.

Jednym z najtrudniejszych problemów do rozwiązania jest kwestia orientacji w terenie. Zagadnienie te zostało pokrótce przedstawione w artykule \cite{threeApproachesToOrientation}.
Artykuł prezentuje również trzy propozycje rozwiązania tego problemu. Warto zauważyć, że każda z tych propozycji znacząco różni się od pozostałych.

\section{Typy istniejących rozwiązań}
Tematyka tworzenia samochodów autonomicznych, bądź też samochodów o dużym stopniu automatyzacji sterowania, stała się w ostatnich latach niezwykle popularna. Z tego powodu istnieje obecnie ogromna ilość projektów nawiązujących do tej tematyki.

Projekty te tworzą bardzo interesujące spektrum przypadków. Różnią między sobą w wielu aspektach, takich jak:
\begin{itemize*}
\item stopień skomplikowania projektu
\item podejście do rozwiązywania poszczególnych aspektów projektu
\item zakres tematyczny projektu
\item środowisko testowania
\item i wiele innych
\end{itemize*}

Na podstawie własnych obserwacji, postanowiłem podzielić te projekty na kilka zasadniczych grup:

\begin{enumerate*}
\item \textbf{Wielkie projekty badawczo-rozwojowe} \\
Projekty o największym stopniu skomplikowania. Rozwijane przez międzynarodowe korporacje posiadające gigantyczny budżet i zaplecze kadrowo-techniczne, potrzebne do realizacji takiego przedsięwzięcia. 
Są to projekty, które do postawionego problemu podchodzą w sposób najbardziej kompleksowy. Ich celem jest stworzenie samochodu poruszającego się w rzeczywistym środowisku, który naprawdę będzie w stanie poradzić sobie w każdych warunkach bez potrzeby ingerencji człowieka.
Część z tych projektów jest rozwijana z myślą o wdrożeniu do produkcji seryjnej, do zastosowania w prawdziwym transporcie drogowym, do przewozu osób na masową skalę.
Inne projekty mają charakter czysto badawczy. Ich celem jest eksploracja nowych strategii rozwiązywania danego problemu. Część z tych strategii, jeśli wykaże swoją wartość podczas serii testów, może zostać zaadoptowana do systemów produkowanych seryjnie. \\

\item \textbf{Środowiska symulacyjne} \\
Nie są to projekty ściśle związane z tworzeniem samochodu autonomicznego. Są to raczej projekty, którego celem jest ułatwienie prac nad rozwojem takich maszyn innym twórcom. Osiągane jest to poprzez tworzenie wirtualnych środowisk symulacyjnych, dostosowanych do tego typu potrzeb.

Praca w symulatorze posiada wiele zalet względem operowania na rzeczywistych maszynach. Najważniejszą z nich jest duża oszczędność pieniędzy i czasu. Symulatory pozwalają na szybkie i bezpieczne testowanie wymyślanych rozwiązań.

Faktem jest, iż wszystkie firmy pracujące nad wielkimi projektami badawczo-rozwojowymi, w mniejszym lub większym stopniu korzystają z takich środowisk symulacyjnych. \\

\item \textbf{Projekty modelarskie} \\
To projekty, które nie są realizowane ani na rzeczywistych samochodach, ani w środowiskach symulacyjnych. Są to projekty, w których eksperymenty przeprowadzane są na miniaturowych modelach samochodów. Czasami te modele nie są odwzorowaniem żadnego rzeczywistego samochodu. Mogą to być projekty wysokobudżetowe, rozwijane w celach komercyjnych. \\

\item \textbf{Małe prywatne projekty} \\
Są to niewielkie projekty, realizowane przez pojedyncze osoby lub małe zespoły złożone z kilku osób. Często podchodzą do problemu w sposób mniej kompleksowy, skupiając się na eksploracji tylko wybranych aspektów problemu. Są to projekty niskobudżetowe, realizowane hobbistycznie, najczęściej w celach edukacyjnych. Rzadko się zdarza, że takie projekty są komercjalizowane. Często są jednak udostępniane publicznie. \\
Najczęściej takie projekty są realizowane w środowiskach symulacyjnych (czasem własnoręcznie tworzonych), rzadziej są to projekty modelarskie. Zdecydowanie najrzadszą, praktycznie niewystępującą grupą są projekty realizowane na rzeczywistych samochodach. Wiąże się to oczywiście z kosztami, jakie należałoby ponieść w takim wypadku.

\end{enumerate*}

Chciałbym w tym miejscu zaznaczyć, że powyższy podział nie może być w żadnym razie traktowany jako formalny. 
Użyte przeze mnie sformułowania mogą nie być precyzyjne, co może prowadzić do nieporozumień i mylnych interpretacji.
Ponadto, klasyfikacja którą przedstawiłem może nie być kompletna. Być może istnieją projekty, których nie można zaklasyfikować do żadnej z powyższych grup. Temat wymaga dalszych badań, co niestety wykracza poza zakres tej pracy. \\
W dalszej części rozdziału zostaną zaprezentowane przykłady dla każdej z opisanych grup. 

\section{Wielkie projekty badawcze}
W obecnych czasach, wiele korporacji o zasięgu międzynarodowym angażuje się w prace nad rozwojem samochodów autonomicznych. 
Niniejsze zestawienie zawiera tylko bardzo mały wycinek z listy istniejących rozwiązań.

\subsection{NVIDIA}
Firma NVIDIA od wielu lat angażuje się w prace badawczo-rozwojowe nad szeroko rozumianą tematyką uczenia maszynowego. Jednym ze szczególnych obszarów zainteresowań firmy jest branża automotive.
NVIDIA nie zajmuje się produkcją samochodów autonomicznych, lecz wytwarzaniem kompleksowych systemów sprzętowo-programowych, które mogą stanowić bazę do tworzenia takich pojazdów. \\
NVIDIA współpracuje z wieloma partnerami, którzy kupują od niej te systemy i wykorzystują do rozwoju własnych rozwiązań. Obecnie jest to ponad 370 partnerów \cite{nvidia:partners}. Wśród nich są między innymi: Toyota, Volkswagen, Audi, Volvo.

Strategia biznesowa, oparta na otwarciu się na współpracę z partnerami działającymi w branży transportowej (i nie tylko), przynosi firmie wymierne korzyści finansowe \cite{nvidia:financialSuccess}. 
Duże zainteresowanie współpracą z firmą NVIDIA jest spowodowane faktem, iż oferowany przez nich system otwiera drogę do zupełnie nowych możliwości. Pozwala między innymi na szybkie wdrożenie usługi przewozów miejskich, o czym przekonała się firma Optimus Ride \cite{nvidia:optimusRide}.

Rozwiązania w zakresie samochodów autonomicznych, oferowane przez firmę NVIDIA, występują pod wspólną nazwą NVIDIA DRIVE \cite{nvidia:drive}. Rozwiązania NVIDIA DRIVE można podzielić na część sprzętową (hardware) i programową (software).

\subsubsection{Rozwiązania sprzętowe - NVIDIA DRIVE AGX}
NVIDIA DRIVE AGX to rodzina platform obliczeniowych, zbudowanych na bazie procesora NVIDIA Xavier.
Obecnie w skład tej rodziny wchodzą dwa modele:
\begin{enumerate*}
\item NVIDIA DRIVE AGX Pegasus \\
Osiąga moc obliczeniową 320 TOPS (Tera Operations Per Second), co czyni go najszybszym rozwiązaniem dostępnym obecnie na rynku. Jego architektura oparta jest na dwóch procesorach NVIDIA Xavier i dwóch kartach graficznych TensorCore. \\
Cechuje się wysoką efektywnością energetyczną. Potrafi utrzymywać w wykonaniu wiele sieci neuronowych jednocześnie. Został zaprojektowany do bezpiecznej obsługi wysoce zautomatyzowanych oraz w pełni autonomicznych systemów jazdy. Kierownica i pedały nie są wymagane.
\begin{figure}[h]
\begin{center}
\includegraphics[width=15cm]{resources/figures/nv-drive-pegasus.jpg}
\caption{NVIDIA DRIVE AGX Pegasus}
\label{NvidiaDrivePegasus}
\end{center}
\end{figure}

\item NVIDIA DRIVE AGX Xavier \\
Jest mniejszy od Pegasusa. Dostarcza jedynie 30 TOPS mocy obliczeniowej. Charakteryzuje się jednak bardzo niskim zużyciem energii, pobiera tylko 30 watów. Przeznaczony dla systemów wspomagania kierowcy. Zbyt słaby dla wsparcia w pełni autonomicznych systemów jazdy.
\begin{figure}[h]
\begin{center}
\includegraphics[width=15cm]{resources/figures/nv-drive-xavier.jpg}
\caption{NVIDIA DRIVE AGX Xavier}
\label{NvidiaDriveXavier}
\end{center}
\end{figure}

\newpage
\item NVIDIA DRIVE Hyperion \\
Najbardziej kompleksowe rozwiązanie oferowane przez firmę NVIDIA. W jego skład wchodzi platforma obliczeniowa (Xavier/Pegasus), zestaw kamer/czujników oraz oprogramowanie sterujące. \\
Więcej informacji o oprogramowaniu znajdzie się w dalszej części rozdziału.
\end{enumerate*}

\subsubsection{Oprogramowanie NVIDIA DRIVE}
W skład pakietu oprogramowania NVIDIA DRIVE wchodzą m.in. następujące komponenty \cite{nvidia:drive:software}:
\begin{enumerate*}
\item DRIVE OS Linux \\
Podstawowy komponent oprogramowania, składający się m.in z wbudowanego systemu czasu rzeczywistego (RTOS), bibliotek NVIDIA CUDA, platformy NVIDIA TensorRT \cite{nvidia:tensorRT} oraz wielu innych modułów. DRIVE OS zapewnia bezpieczne środowisko uruchomieniowe dla aplikacji. Pełni funkcję systemu operacyjnego dla platform obliczeniowych z rodziny NVIDIA DRIVE AGX.
\item DRIVE AV \\
Dostarcza modułów dla percepcji otoczenia (NVIDIA DRIVE Perception \cite{nvidia:drivePerception}), mapowania środowiska (NVIDIA DRIVE Mapping \cite{nvidia:driveMapping}) i planowania trasy (NVIDIA DRIVE Planning \cite{nvidia:drivePlanning})
\item DRIVE IX \\
Dostarcza algorytmów do wizualizacji otoczenia pojazdu, monitoruje zachowanie kierowcy i służy jako asystent kabiny.
\item DriveWorks \\
Jest frameworkiem dostarczającym zestaw bibliotek i narzędzi dla użytkowników pakietu NVIDIA DRIVE.
\end{enumerate*}
\begin{figure}[h]
\begin{center}
\includegraphics[width=15cm]{resources/figures/nvidia-drive-software-stack.jpg}
\caption{Oprogramowanie wchodzące w skład pakietu NVIDIA DRIVE}
\label{NvidiaDriveSoftwareStack}
\end{center}
\end{figure}

\subsection{Udacity}
Udacity jest edukacyjną organizacją oferującą tzw. ,,masywne otwarte kursy online'' (MOOCs - Massive Open Online Courses) \cite{mooc:guide}. Posiadają własną platformę kursową, na której publikowane są kursy dotyczące szeroko pojętej branży IT. \\
Jednym z kursów, oferowanych przez Udacity, jest \textit{Self-Driving Car Engineer Nanodegree program}. Został stworzony w partnerstwie z gigantami branż automotive oraz IT, m.in. Mercedes-Benz, NVIDIA, Uber, BMW.

Wraz z uruchomieniem kursu, firma Udacity zdecydowała się na dosyć odważny krok. Postanowiła zrealizować, we współpracy z programistami z całego świata, własny projekt samochodu autonomicznego. Projekt ten ma być w całości \textbf{otwartoźródłowy}. \\
Jak opisuje Oliver Cameron, ówczesny lider programu Udacity Self-Driving Car \cite{udacity:selfDrivingCar}: \\
,,Gdy decydowaliśmy się na tworzenie programu  \textit{Self-Driving Car Engineer}, od razu wiedzieliśmy, że musimy zbudować własny samochód autonomiczny. (...) utworzyliśmy więc zespół \textit{Self-Driving Car Team}. Jedna z pierwszych decyzji jaką podjęliśmy? \textbf{Kod open source, pisany przez setki studentów z całego świata!}''

Aby wcielić ten projekt w życie, firma Udacity dokonała niezbędnych przygotowań:
\begin{enumerate*}
\item Zakup samochodu marki Lincoln MKZ (rocznik 2016)
\item Montaż dodatkowych komponentów:
\begin{itemize*}
\item Dwa radary LIDAR marki Velodyne VLP-16
\item Jeden radar marki Delphi
\item Trzy kamery marki Point Grey Blackfly
\item Xsens IMU \cite{imu:wikipedia}
\item ECU \cite{ecu:wikipedia}
\item i wiele innych
\end{itemize*}
\item Konfiguracja systemu ROS (Robot Operating System) \cite{ros:about}
\item Stworzenie wstępnej bazy kodu
\end{enumerate*}


\subsection{Waymo}
\section{Środowiska symulacyjne}
\subsection{Microsoft AirSim}
\subsection{Voyage Deepdrive}
\section{Projekty modelarskie}
\subsection{donkeycar}
\section{Małe prywatne projekty}
\subsection{Deep Learning Cars}
Autorem projektu jest austriacki programista Samuel Arzt \cite{artz:deepLearningCars:blog} \cite{artz:deepLearningCars:github}. \\

\begin{figure}[h]
\begin{center}
\includegraphics[width=16cm]{resources/figures/samuelArtzDeepCars.png}
\caption{Stopklatka z pracy systemu}
\label{SamuelArtzDeepLearningCars}
\end{center}
\end{figure}

\subsubsection{Opis implementacji}
Na początku każdej generacji jest tworzonych 20 samochodów (osobników). \\
Każdy samochód posiada własną sieć neuronową oraz zestaw 5 czujników, odmierzających odległość z przodu samochodu.

Sygnały z czujników wysyłane są do neuronów warstwy wejściowej.
Czujniki te służą do wykrywania przeszkód w otoczeniu samochodu.
Każdy czujnik jest skierowany w inną stronę.
Czujniki tworzą "pole widzenia" samochodu o łącznym kącie 90 stopni.

Zadaniem każdego samochodu jest bezkolizyjny przejazd po wyznaczonym torze.
Jedyne operacje jakie może wykonać samochód, to skręt w lewo lub prawo.
Prędkość wszystkich samochodów jest stała i ustalana przed rozpoczęciem generacji.
W chwili kolizji z przeszkodą, samochód umiera.
Należy dodać, iż nie występują kolizje pomiędzy samochodami.

Kiedy wszystkie samochody umrą, bieżąca generacja się kończy.
Osobniki do kolejnej generacji są tworzone przy pomocy algorytmu ewolucyjnego.

Do reprodukcji wybierane są 2 osobniki z najwyższym przystosowaniem.

Przystosowanie jest obliczane na podstawie odległości przebytej przez samochód.
Im większa przejechana odległość, tym wyższa wartość przystosowania samochodu.

Geny dwóch najlepiej przystosowanych osobników są krzyżowane i mutowane, tworząc 20 nowych osobników (potomków).

Genami samochodu są wagi w jego sieci neuronowej.
Zastosowane sieci neuronowe to sieci jednokierunkowe (tzw. feedforward),
z pięcioma neuronami w warstwie wejściowej, dwoma neuronami w warstwie wyjściowej i dwiema warstwami ukrytymi (odpowiednio 4 i 3 neurony).
Uczenie sieci (dostrajanie wartości wag) odbywa się za pomocą algorytmu genetycznego \cite{geneticAlgorithm:introduction}.

\subsubsection{Reprezentacja graficzna}
Dwuwymiarowa symulacja środowiska (toru) została wykonana przy użyciu silnika Unity.
Krzyżyki wyświetlane na ekranie są reprezentacją czujników samochodów.
Kamera podąża za samochodem z najlepszym przystosowaniem.
Dwa samochody z najlepszym przystosowaniem są wyróżniane kolorem.

\subsubsection{Dostępność kodu źródłowego}
Cały kod źródłowy projektu został udostępniony przez autora. \\
Link do repozytorium znajduje się w pozycji literaturowej: \cite{artz:deepLearningCars:github}.
\chapter{Wstęp do sztucznych sieci neuronowych}
\section{Pojęcia podstawowe}
\subsection{Sztuczne sieci neuronowe}
\subsubsection{Czym są sztuczne sieci neuronowe?}
Sztuczna sieć neuronowa jest modelem obliczeniowym, którego filozofia pracy oparta jest na funkcjonowaniu biologicznego układu nerwowego.
Sztuczna sieć neuronowa składa się z dużej liczby wzajemnie ze sobą powiązanych elementów, zwanych neuronami.

Sieci neuronowe są wykorzystywane w informatyce do rozwiązywania problemów, z którymi nie radzą sobie tradycyjne modele obliczeniowe.

Najważniejszą cechą sieci neuronowych jest ich zdolność uczenia się, w oparciu o dostarczany im zbiór danych treningowych.

\subsubsection{Sieci neuronowe kontra tradycyjne modele obliczeniowe}
Sieci neuronowe w sposób zasadniczy różnią się od tradycyjnych modeli obliczeniowych.
 
W tradycyjnym modelu obliczeniowym, wszystkie kroki wykonywanego algorytmu muszą być ściśle zdefiniowane przez programistę.

Natomiast w przypadku sieci neuronowych, algorytm rozwiązywanego problemu może być dla programisty nieznany lub znany tylko na bardzo ogólnym poziomie.
Sieć neuronowa jest w stanie samodzielnie odnaleźć optymalny algorytm rozwiązania.

\subsubsection{Czym jest sztuczny neuron?}
Sztuczny neuron jest matematyczną funkcją, posiadającą wiele wejść i tylko jedno wyjście. Danymi na wejściu i wyjściu neuronu są najczęściej liczby rzeczywiste, będące  w zadanym zakresie.

Przyjęło się, aby każde wejście neuronu miało przypisaną wagę, czyli wartość liczbową określającą jak ważne jest dane wejście dla neuronu.
Bardzo często wartości wag mieszczą się w przedziale [-1; 1].

Sztuczny neuron posiada 2 tryby pracy - tryb nauki oraz tryb odtwarzania. \newline
W trybie nauki neuron może być uczony, aby na odpowiedni zestaw danych
wejściowych generował na wyjściu określoną wartość liczbową. \newline
W trybie odtwarzania, neuron klasyfikuje zestawy danych wejściowych, generując
odpowiadające im wyjścia (zgodnie z tym, czego nauczył się w trybie uczenia).

\subsubsection{Budowa neuronu biologicznego}
Ponieważ budowa sztucznego neuronu opiera się na budowie naturalnej komórki nerwowej, dlatego ważne aby zrozumieć jak neuron biologiczny jest zbudowany. \\
Najważniejsze jego elementy to:
\begin{itemize*}
\item Jądro - centrum obliczeniowe neuronu.
\item Dendryty - wejścia neuronu. Przesyłają do jądra sygnały poddawane późniejszej obróbce.
\item Synapsa - łączy dendryt z jądrem. W synapsie sygnał wejściowy może ulegać wstępnej modyfikacji, to znaczy być wzmacniany lub osłabiany.
\item Wzgórek aksonu - łączy jądro z aksonem.
\item Akson - wyjście neuronu. Zazwyczaj rozgałęzia się, przesyłając sygnał do wielu wejść kolejnych neuronów.
\end{itemize*}

\begin{figure}[h]
\begin{center}
\includegraphics[width=10cm]{resources/figures/natural_neuron.png}
\caption{Model budowy neuronu biologicznego}
\end{center}
\end{figure}

\newpage
\subsubsection{Budowa sztucznego neuronu}
Porównując budowę sztucznego neuronu z neuronem biologicznym, można znaleźć wiele analogii. \\
Koncepcyjny model budowy sztucznego neuronu przedstawiono na rysunku poniżej.

\vspace{1cm}
\begin{figure}[h]
\begin{center}
\includegraphics[width=15cm]{resources/figures/artificial_neuron.png}
\caption{Model budowy sztucznego neuronu}
\end{center}
\end{figure}
\newpage

\section{Metody uczenia sztucznych sieci neuronowych}
\chapter{Projekt systemu}

\section{Analiza wymagań}
\subsection{Studium możliwości}
\subsection{Wymagania funkcjonalne}
\subsection{Ograniczenia projektu}

\section{Projekt}
\subsection{Projekt warstwy danych}
\subsection{Projekt warstwy logiki}
\subsection{Projekt warstwy interfejsu użytkownika}
\chapter{Zastosowane technologie}
\section{Język programowania}
\section{Biblioteki}
\chapter{Opis implementacji}
\section{Implementacja: punkty kluczowe}
\section{Testy systemu}
\chapter{Eksperymenty obliczeniowe}
\label{ExperimentsChapter}

W niniejszym rozdziale przedstawiam praktyczne aspekty tworzonej pracy, czyli eksperymenty obliczeniowe wykonane przy użyciu stworzonej przeze mnie aplikacji. Analizę wyników poprzedzam omówieniem metodyki, jaką stosowałem podczas przeprowadzania eksperymentów.

\section{Metodyka eksperymentów}
Metodyka eksperymentów obliczeniowych to lista przyjętych założeń dla przeprowadzanych eksperymentów. Oto najważniejsze założenia:
\begin{enumerate*}
\item Wszystkie eksperymenty muszą wykonać się automatycznie, bez potrzeby aktywnego udziału człowieka. Pierwszy eksperyment powinien rozpocząć się po wywołaniu jednej prostej komendy w linii poleceń. Komendą jest nazwa skryptu implementującego główną logikę przebiegu eksperymentów.
\item Parametry dla eksperymentów obliczeniowych muszą być dostarczane z zewnętrznego pliku konfiguracyjnego o ustalonym formacie. Ścieżka do pliku konfiguracyjnego musi być podawana jako argument linii poleceń.
\item Liczba prób musi być podana jako argument linii poleceń. Jedna próba to wykonanie pełnego zestawu eksperymentów. Im więcej prób, tym bardziej wiarygodne wyniki końcowe. Z drugiej strony, zwiększanie liczby prób znacząco wydłuża czas wykonania kompletnej serii eksperymentów.
\item Podczas wykonywania eksperymentów, jak najwięcej istotnych informacji powinno być zapisywanych do pliku dziennika (zwanego również \textit{logiem}). W przypadku problemów z aplikacją, posiadanie logów potrafi znacząco usprawnić proces debugowania.
\item Brak zainstalowanego silnika Unity nie powinien uniemożliwiać przeprowadzenia eksperymentów obliczeniowych.
\item Wynikiem końcowym powinien być zestaw wykresów, utworzonych na bazie informacji zgromadzonych podczas przebiegu eksperymentów. Wykresy powinny dotyczyć istotnych danych, pozwalających na wyciągnięcie konstruktywnych wniosków.
\item Podczas każdej próby przeprowadzanych jest sześć eksperymentów. Każdy eksperyment to trening populacji na jednym z trzech torów wyścigowych, przy użyciu jednego z dwóch dostępnych algorytmów uczenia - \textbf{Ewolucji Różnicowej} (patrz sekcja \ref{DeOverview}) lub \textbf{PSO} (patrz sekcja \ref{PsoOverview}). Po każdym treningu, najlepiej wyuczony model jest ewaluowany na wszystkich torach wyścigowych. Ewaluacja ma na celu sprawdzenie, czy model wytrenowany na danym torze poradzi sobie z pozostałymi torami wyścigowymi.
\end{enumerate*}

\section{Opis implementacji}
Główna logika przebiegu eksperymentów została zaimplementowana w skrypcie \textit{experiment.py}. W celu rozpoczęcia eksperymentów, ten skrypt należy wywołać z linii poleceń. Oto definicja API linii poleceń:

\begin{minted}[ fontsize=\fontsize{10}{9} ] {python}
def getProgramOptions():
    APP_USAGE_DESCRIPTION = """
Run series of experiments which result in generating charts for IT Engineering Thesis.
NOTE: As a config file should be used 'config.json' file or other with appropriate
fields.

Usage:
    experiment.py <config-file-path> [options]
    experiment.py -h | --help

Options:
    --num-of-trials=<n>      Specify number of trials used to generate data.
                             [default: 10]
    -v --verbose             Run in verbose mode.
"""
    options = docopt(APP_USAGE_DESCRIPTION)
    return options
\end{minted}

Wywołanie skryptu \textit{experiment.py} wymaga podania co najmniej jednego argumentu. Jest nim ścieżka do pliku konfiguracyjnego (patrz sekcja \ref{ConfigOverview}). Pozostałe dwa argumenty są opcjonalne. Pierwszym z nich jest liczba prób (domyślną wartością jest 10), natomiast drugi parametr jest flagą, której ustawienie decyduje o tym, czy dane zapisywane do dziennika powinny być również wyświetlane w oknie terminala.

Po przetworzeniu argumentów linii poleceń, skrypt \textit{experiment.py} wykonuje kilka czynności mających przygotować grunt pod rozpoczęcie pętli eksperymentów. Wśród tych czynności jest między innymi wczytanie pliku konfiguracyjnego oraz inicjalizacja wymaganych zmiennych.

Po tym etapie następuje wykonanie pętli eksperymentów. Implementacja pętli wygląda następująco:
\begin{minted}[ fontsize=\fontsize{10}{9} ] {python}
# --- Experiment sequence loop --- #
    for trialCounter in range(numberOfTrials):
        for trackNumber in range(1, 4):
            experimentLog.Append("Generating data from 'train_de.py', track: " \
                    "{0}, trial: {1}".format(trackNumber, trialCounter + 1))
            generateDataFromTraining(
                    train_de,
                    "DE",
                    trackNumber,
                    pathToConfigFile,
                    buildPaths,
                    experimentLog,
                    dataCollector,
                    minFitnessDict,
                    isVerbose = options["--verbose"])
            
            experimentLog.Append("Generating data from 'train_pso.py', track: " \
                    "{0}, trial: {1}".format(trackNumber, trialCounter + 1))
            generateDataFromTraining(
                    train_pso,
                    "PSO",
                    trackNumber,
                    pathToConfigFile,
                    buildPaths,
                    experimentLog,
                    dataCollector,
                    minFitnessDict,
                    isVerbose = options["--verbose"])
\end{minted}

Zewnętrzna pętla wykonuje się tyle razy, ile wynosi liczba prób. Dla każdej próby trening odbywa się na wszystkich trzech torach oferowanych przez obecne Środowisko Uczenia. Treningi odbywają się przy użyciu dwóch algorytmów - \textbf{Ewolucji Różnicowej} oraz \textbf{PSO}. Najważniejszym elementem powyższego kodu jest funkcja \textit{generateDataFromTraining}, która przeprowadza treningi oraz dokonuje walidacji wytrenowanych modeli.

\section{Analiza uzyskanych wyników}
Eksperymenty zostały wykonane na stacji roboczej o specyfikacji opisanej w sekcji \ref{HardwareSpecs}. Liczba prób wyniosła 30. Łączny czas obliczeń wyniósł 16 godzin, 23 minuty i 36 sekund, co daje średni czas obliczeń dla jednej próby na poziomie 32 minut i 47 sekund. \\
Parametry algorytmów uczących miały następujące wartości:
\begin{enumerate*}
\item \textbf{Parametry dla Ewolucji Różnicowej}
\begin{itemize*}
\item Rozmiar populacji: 50 osobników
\item Współczynnik mutacji: 0.8
\item Prawdopodobieństwo krzyżowania: 0.7
\end{itemize*}
\item \textbf{Parametry dla algorytmu PSO}
\begin{itemize*}
\item Rozmiar populacji: 100 osobników
\item W = 0.729, $c_1 = 2.05$, $c_2 = 2.05$
\end{itemize*}
\end{enumerate*}

\subsection{Liczba wygenerowanych rozwiązań}
Rysunek \ref{SearchCount} przedstawia średnie liczby wygenerowanych rozwiązań kandydackich przed odnalezieniem rozwiązania właściwego. Są to średnie liczone ze wszystkich przeprowadzonych prób. Każda średnia dotyczy danego toru i danego algorytmu uczącego. Im średnia jest mniejsza, tym szybciej model został wytrenowany. Wykres na rysunku \ref{SearchCount} dokonuje porównania algorytmów uczących (czyli Ewolucji Różnicowej oraz PSO) dla poszczególnych torów.

\vspace{1cm}
\begin{figure}[H]
\centering
\includegraphics[width=15cm]{resources/figures/search_count.png}
\caption{Średnia liczba wygenerowanych rozwiązań kandydackich}
\label{SearchCount}
\end{figure}

Z wykresu można wysnuć następujące obserwacje:
\begin{enumerate*}
\item Liczba wygenerowanych rozwiązań rośnie wraz ze wzrostem poziomu trudności każdego toru. Tor \textit{RaceTrack}\_3 jest znacznie trudniejszy od pozostałych, dlatego wymaga znacznie więcej rozwiązań kandydackich zanim właściwe rozwiązanie zostanie odnalezione.
\item Na torze \textit{RaceTrack}\_1 obydwa algorytmy radzą sobie porównywalnie. Niewielką przewagę posiada Ewolucja Różnicowa, która okazała się lepsza o zaledwie 7\% od algorytmu PSO.
\item Na torze \textit{RaceTrack}\_2 Ewolucja Różnicowa również jest lepsza, ale przewaga nad PSO jest znacznie większa. Wynosi aż 39\%.
\item Wyniki uzyskane dla toru \textit{RaceTrack}\_3 mogą trochę zaskakiwać, zwłaszcza biorąc pod uwagę wyniki z pozostałych torów. W tym przypadku, \textbf{Ewolucja Różnicowa jest znacznie gorsza od PSO}, a przewaga algorytmu PSO wynosi 29\%.
\end{enumerate*}

Z powyższych obserwacji wynika, że pewne cechy torów wyścigowych ,,faworyzują'' dany algorytm. Natomiast bardzo trudno jest w tej chwili wskazać, jakie to są cechy. Uzyskanie odpowiedzi na to pytanie wymagałoby znacznie głębszej analizy tematu, co wykracza poza zakres pracy.

\vspace{2cm}
\subsection{Czas treningu}
Rysunek \ref{MeanTimeSeconds} przedstawia średnie czasy treningu liczone w sekundach. Czasy te są średnimi ze wszystkich prób. Każdy czas dotyczy treningu na konkretnym torze i przy użyciu konkretnego algorytmu uczącego. Obserwacje jakie można wysnuć z tego wykresu są następujące:
\begin{enumerate*}
\item Czas treningu wzrasta wraz ze wzrostem poziomu trudności każdego toru. Tor \textit{RaceTrack}\_3 jest znacznie trudniejszy od pozostałych, dlatego wymaga znacznie więcej czasu na trening.
\item Ewolucja Różnicowa ma lepszy czas treningu na wszystkich torach wyścigowych, choć dla pierwszego i trzeciego toru różnice te nie są wielkie. Natomiast w przypadku toru \textit{RaceTrack}\_2 różnica w czasie treningu jest znacznie większa i wynosi aż 45\%.
\end{enumerate*}
\vspace{1cm}
\begin{figure}[H]
\centering
\includegraphics[width=15cm]{resources/figures/train_time_seconds.png}
\caption{Średni czas treningu w sekundach}
\label{MeanTimeSeconds}
\end{figure}

Z kolei na rysunku \ref{MeanTimeEpisodes} zamieszczono średnie czasy treningu liczone w epizodach. Czasy te dla torów \textit{RaceTrack}\_1 oraz \textit{RaceTrack}\_2 są takie same, natomiast przy torze \textit{RaceTrack}\_3 można zauważyć spory rozstrzał pomiędzy wynikami dla poszczególnych algorytmów. Liczba epizodów potrzebna do wytrenowania modelu przy użyciu algorytmu PSO jest o \textbf{ponad połowę mniejsza} od liczby epizodów wymaganych przy Ewolucji Różnicowej. Biorąc jednak pod uwagę rysunek \ref{MeanTimeSeconds}, każdy epizod treningu przy użyciu Ewolucji Różnicowej liczył się średnio o wiele szybciej od analogicznego epizodu dla algorytmu PSO.

\subsection{Walidacja wyuczonych modeli}
Rysunki \ref{ValidationDE} oraz \ref{ValidationPSO} przedstawiają wyniki walidacji modelów wytrenowanych obydwoma algorytmami uczącymi. Walidacja polega na sprawdzeniu wytrenowanego modelu na wszystkich trzech torach. Walidacja występuje po zakończeniu każdego treningu. Liczby ponad słupkami oznaczają liczbę walidacji zakończonych sukcesem. Maksymalna liczba pozytywnych walidacji dla danego przypadku jest równa liczbie prób. W omawianym przypadku będzie to więc 30.
%\vspace{0.4cm}
\begin{figure}[H]
\centering
\includegraphics[width=15cm]{resources/figures/train_time_episodes.png}
\caption{Średni czas treningu w epizodach}
\label{MeanTimeEpisodes}
\end{figure}

Etykiety na dole wykresów (\textit{RaceTrack}\_1, \textit{RaceTrack}\_2, \textit{RaceTrack}\_3) wyznaczają, na jakim torze modele były trenowane. Kolory słupków wyznaczają, na jakim torze modele były walidowane.

Obserwacje jakie można wysnuć z rysunków są następujące:
\begin{enumerate*}
\item Modele wytrenowane na łatwiejszych torach rzadko przechodzą walidację na torach trudniejszych;
\item Modele wytrenowane na trudniejszych torach zazwyczaj dobrze radzą sobie z torami łatwiejszymi;
\item Ponieważ tor \textit{RaceTrack}\_3 jest o wiele trudniejszy do wyuczenia się niż pozostałe tory, dlatego też tylko modele wytrenowane na tym torze potrafiły być tam pozytywnie walidowane. Wyjątkiem jest jeden model, wytrenowany na torze \textit{RaceTrack}\_2 przy użyciu Ewolucji Różnicowej.
\item W przypadku algorytmu PSO, dwukrotnie doszło do sytuacji w której model trenowany na torze \textit{RaceTrack}\_3 nie zdążył się wytrenować na tyle dobrze, żeby móc pokonać choćby najprostszy z torów wyścigowych. Ten fakt pokazuje, że trening na trudniejszym torze utrudnia również wytrenowanie modelu radzącego sobie na torach łatwiejszych.
\item Ponieważ tor \textit{RaceTrack}\_2 nie jest dużo trudniejszy od toru \textit{RaceTrack}\_1, dlatego część modeli wytrenowanych na torze \textit{RaceTrack}\_1 radziła sobie także na torze \textit{RaceTrack}\_2. W przypadku Ewolucji Różnicowej współczynnik ten wynosił 30\%, natomiast w przypadku algorytmu PSO było to 20\%.
\end{enumerate*}
\vspace{0.5cm}
\begin{figure}[H]
\centering
\includegraphics[width=15cm]{resources/figures/validation_de.png}
\caption{Walidacja modeli wytrenowanych Ewolucją Różnicową}
\label{ValidationDE}
\end{figure}

\section{Wnioski z analiz}
Analiza uzyskanych wyników pozwala na empiryczne potwierdzenie faktu zgodnego z intuicją. Im trudniejszy tor, tym więcej czasu oraz obliczeń jest potrzebnych do wyuczenia na nim modelu. Statystyki dla torów \textit{RaceTrack}\_1 oraz \textit{RaceTrack}\_2 są zbliżone, ponieważ poziom trudności tych dwóch torów jest do siebie zbliżony. Natomiast statystyki dla toru \textit{RaceTrack}\_3 znacznie odbiegają od reszty, ponieważ jest on dużo trudniejszy od pozostałych torów.

\vspace{0.5cm}
\begin{figure}[H]
\centering
\includegraphics[width=15cm]{resources/figures/validation_pso.png}
\caption{Walidacja modeli wytrenowanych algorytmem PSO}
\label{ValidationPSO}
\end{figure}

Uzyskane wyniki pozwalają dowiedzieć się czegoś o wpływie Środowiska Uczenia na proces treningu. Niewiele natomiast mówią o samych algorytmach uczących. Wyciągnięcie bardziej wartościowych wniosków na ten temat wymagałoby dalszych, dogłębnych badań.

\section{Analiza wytrenowanego modelu}
Rysunek \ref{TrainedNetworkExample} przedstawia wizualizację sieci neuronowej jednego z wyuczonych modeli. Jest to model wyuczony na torze \textit{RaceTrack}\_3. Na rysunku zostały zobrazowane parametry sieci. Wartości liczbowe zawarte wewnątrz neuronów to ich biasy, natomiast liczby znajdujące się przy krawędziach to wagi poszczególnych połączeń.

Na podstawie bezpośrednich obserwacji rysunku trudno wyciągnąć wartościowe wnioski, natomiast pewien wgląd na ,,strategię działania'' zakodowaną w parametrach sieci daje nam rozważenie kilku scenariuszy testowych, czyli sytuacji które mogą się zdarzyć podczas nawigowania samochodem po Środowisku Uczenia. Tabela \ref{InputOutputExamples} przedstawia wyniki obliczone przez sieć neuronową dla wybranych danych wejściowych. Wartości danych wejściowych zostały dobrane pod kątem rozpatrzenia podstawowych scenariuszy testowych.

\begin{table}[]
\centering
\begin{tabular}{|c|c|c|c|}
\hline
\textbf{Dane wejściowe} & \textbf{\begin{tabular}[c]{@{}c@{}}Wyniki\\ obliczeń sieci\end{tabular}} & \textbf{Scenariusz testowy}                                                                       & \textbf{Zachowanie samochodu}                                                \\ \hline
{[}0.5, 1.0, 0.5{]}     & {[}1.0, 1.0{]}                                                           & \begin{tabular}[c]{@{}c@{}}Prosta szeroka droga,\\ samochód na środku drogi\end{tabular}          & \begin{tabular}[c]{@{}c@{}}Wyrównuje do\\ prawej krawędzi drogi\end{tabular} \\ \hline
{[}0.2, 1.0, 0.2{]}     & {[}0.81, 0.24{]}                                                         & \begin{tabular}[c]{@{}c@{}}Prosta wąska droga,\\ samochód na środku drogi\end{tabular}            & \begin{tabular}[c]{@{}c@{}}Wyrównuje do\\ prawej krawędzi drogi\end{tabular} \\ \hline
{[}1.0, 1.0, 0.3256{]}  & {[}1.0, 0.0{]}                                                           & \begin{tabular}[c]{@{}c@{}}Prosta szeroka droga,\\ samochód blisko\\ prawej krawędzi\end{tabular} & Jedzie prosto                                                                \\ \hline
{[}1.0, 0.2, 0.3{]}     & {[}1.0, -0.82{]}                                                         & Zakręt w lewo                                                                                     & Skręca w lewo                                                                \\ \hline
{[}0.3, 0.2, 1.0{]}     & {[}1.0, 1.0{]}                                                           & Zakręt w prawo                                                                                    & Skręca w prawo                                                               \\ \hline
\end{tabular}
\caption{Zachowanie sieci dla wybranych scenariuszy drogowych}
\label{InputOutputExamples}
\end{table}

Na podstawie zawartości tabeli \ref{InputOutputExamples} można wysnuć wniosek, że wytrenowana sieć generuje poprawne wyniki. Potrafi skręcać we właściwą stronę podczas zakrętu. Potrafi też jechać prosto gdy tego wymaga sytuacja. Jedyną ciekawostką jest fakt, że omawiana sieć preferuje ,,trzymania się'' blisko prawej krawędzi drogi. Taka właściwość sieci została nabyta podczas procesu uczenia się.

\vspace{1.5cm}
\begin{figure}[H]
\centering
\includegraphics[width=15cm]{resources/figures/trained_model_example.png}
\caption{Przykład wytrenowanej sieci neuronowej}
\label{TrainedNetworkExample}
\end{figure}
\chapter*{Podsumowanie}
\addcontentsline{toc}{chapter}{Podsumowanie}

Celem pracy było opracowanie prostego systemu uczącego sieci neuronowe, bazującego na symulacjach przeprowadzanych w wymodelowanym środowisku. Zadaniem sieci neuronowych było sterowanie samochodem, poruszającym się po wirtualnym środowisku symulacyjnym. Sieci neuronowe były uczone za pomocą algorytmów PSO i Ewolucji Różnicowej.

Cel pracy został zrealizowany w całości. Stworzona aplikacja umożliwia przeprowadzanie treningu sieci neuronowych. Ewolucja Różnicowa (patrz sekcja \ref{DeOverview}) oraz algorytm PSO (patrz sekcja \ref{PsoOverview}) były z sukcesami zastosowane do uczenia sieci. Wytrenowane sieci neuronowe potrafią pokonywać wymodelowane tory wyścigowe (patrz sekcja \ref{UnityScenes}), nawet te o dużym stopniu skomplikowania. Więcej informacji o wynikach uczenia sieci można odnaleźć w rozdziale \ref{ExperimentsChapter}.

\vspace{2cm}
\section*{Perspektywy dalszych badań w dziedzinie}
Działania przeprowadzone w ramach tworzenia niniejszej pracy stanowią zaledwie wstęp do badania omawianego zagadnienia. Uczenie maszynowe oraz tworzenie autonomicznych samochodów to dwa bardzo szerokie tematy, którym można poświęcić wiele lat badań.

Jednym z możliwych kierunków dalszego rozwoju byłoby przeprowadzenie eksperymentów na pojeździe poruszającym się po rzeczywistym środowisku. Takim pojazdem mógłby być model samochodu, wykonany w pewnej skali. Innym pomysłem wartym rozważenia jest poszerzenie percepcji samochodu poprzez dołączenie mechanizmu wizji komputerowej, o której opowiada artykuł \cite{computerVision:overview}. Wykorzystanie tej technologii mogłoby w znaczący sposób zwiększyć możliwości tworzonego pojazdu.

\newpage
\section*{Opinie i przemyślenia}
Tematyka poruszana w tej pracy jest bardzo ważna i istotna, nie tylko dla osób zainteresowanych nowinkami technicznymi. Uczenie maszynowe oraz szerzej pojęta sztuczna inteligencja zmienią sposób, w jaki będzie funkcjonować społeczeństwo przyszłości. Będą miały istotny wpływ na życie każdego człowieka. 

W ciągu najbliższych kilkunastu lat czeka nas wielka rewolucja przemysłowa, w wyniku której zniknie wiele wykonywanych obecnie zawodów \cite{czwartaRewolucja:artykul}. W wyniku tego, sporo ludzi będzie zmuszonych do zmiany pracy.

Istotnym tematem jest także kwestia odpowiedzialnego wykorzystania zdobyczy technologicznych. Wiele osób, uznawanych za wybitne w świecie nauki, ostrzegają przed skutkami nieetycznego wykorzystania sztucznej inteligencji. Do takich osób należy m.in. Elon Musk, który wielokrotnie wypowiadał się na ten temat \cite{elonMusk:sztucznaInteligencja}. Bardzo ważne, aby wszystkie osoby zajmujące się tą tematyką miały świadomość odpowiedzialności, jaka na nich spoczywa. Dotyczy to zwłaszcza polityków, uczestniczących w procesie tworzenia dokumentów legislacyjnych. Odpowiednie regulacje prawne z zakresu sztucznej inteligencji są niezbędne. Tylko dzięki nim jest szansa, że zwykły obywatel nie będzie czuł zagrożenia w sytuacji, w której nowoczesne technologie coraz bardziej wkraczają w jego życie codzienne.

Musimy zapewnić, aby sztuczna inteligencja pozostała narzędziem, które jest w naszej kontroli i które nie jest wykorzystywane do celów nieetycznych. Nie możemy dać się zaślepić możliwościom, jakie oferuje nam ta technologia. Nie możemy bagatelizować potencjalnych zagrożeń. Tylko dzięki wypracowaniu odpowiednich zabezpieczeń jest szansa, że sztuczna inteligencja nigdy nie odwróci się przeciwko ludzkości.

\addcontentsline{toc}{chapter}{Spis rysunków} 
\listoffigures

%   this is for BibTeX
\bibliographystyle{plplain}
\bibliography{literatura}

%adds the bibliography to the table of contents
\addcontentsline{toc}{chapter}
         {\protect\numberline{Bibliografia\hspace{-96pt}}}
\end{document}