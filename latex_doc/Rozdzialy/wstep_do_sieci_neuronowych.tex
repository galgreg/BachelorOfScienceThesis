\chapter{Wstęp do sztucznych sieci neuronowych}
\section{Pojęcia podstawowe}
\subsection{Sztuczne sieci neuronowe}
\subsubsection{Czym są sztuczne sieci neuronowe?}
Sztuczna sieć neuronowa jest matematycznym modelem, wzorowanym na pracy biologicznego układu nerwowego.
Składa się z dużej liczby wzajemnie ze sobą powiązanych elementów,
zwanych neuronami.

Sieci neuronowe w sposób zasadniczy różnią się od tradycyjnych systemów informatycznych. 
W tradycyjnych systemach informatycznych, każdy element systemu musi być ściśle określony oraz zaprogramowany:
\begin{itemize}
\item bezpośrednio - przez twórców systemu
\item pośrednio - przez dostawców bibliotek wykorzystywanych w systemie
\end{itemize}

Sieci neuronowe są projektowane w odmienny sposób.
Poprawne funkcjowanie sieci neuronowych jest osiągane poprzez zdefiniowanie jej topologii oraz najważniejszych parametrów (takich jak funkcja aktywacji oraz algorytm uczący). \\
Natomiast sam algorytm wykonania zadań zleconych sieciom neuronowym może być dla programisty nieznany. \\
Sieć neuronowa jest w stanie samodzielnie opracować sobie taki algorytm, dysponując odpowiednio wielkim zbiorem danych wejściowych oraz (opcjonalnie) zbiorem oczekiwanych rezultatów.

\section{Metody uczenia sztucznych sieci neuronowych}