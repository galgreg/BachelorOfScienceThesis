\chapter{Wstęp do sztucznych sieci neuronowych}
\section{Pojęcia podstawowe}
\subsection{Czym są sztuczne sieci neuronowe?}
Sztuczna sieć neuronowa jest modelem obliczeniowym, którego filozofia pracy oparta jest na funkcjonowaniu biologicznego układu nerwowego.
Sztuczna sieć neuronowa składa się z dużej liczby wzajemnie ze sobą powiązanych elementów, zwanych neuronami.

Sieci neuronowe są wykorzystywane w informatyce do rozwiązywania problemów, z którymi nie radzą sobie tradycyjne modele obliczeniowe.

Najważniejszą cechą sieci neuronowych jest ich zdolność uczenia się, w oparciu o dostarczany im zbiór danych treningowych.

\subsection{Czym jest sztuczny neuron?}
Sztuczny neuron jest matematyczną funkcją, posiadającą wiele wejść i tylko jedno wyjście. Danymi na wejściu i wyjściu neuronu są najczęściej liczby rzeczywiste, będące  w zadanym zakresie.

Przyjęło się, aby każde wejście neuronu miało przypisaną wagę, czyli wartość liczbową określającą jak ważne jest dane wejście dla neuronu.
Bardzo często wartości wag mieszczą się w przedziale [-1; 1].

Sztuczny neuron posiada 2 tryby pracy - tryb nauki oraz tryb odtwarzania. \newline
W trybie nauki neuron może być uczony, aby na odpowiedni zestaw danych
wejściowych generował na wyjściu określoną wartość liczbową. \newline
W trybie odtwarzania, neuron klasyfikuje zestawy danych wejściowych, generując
odpowiadające im wyjścia (zgodnie z tym, czego nauczył się w trybie uczenia).

\subsection{Budowa neuronu biologicznego}
Ponieważ budowa sztucznego neuronu opiera się na budowie naturalnej komórki nerwowej, dlatego ważne jest, żeby zrozumieć jak neuron biologiczny jest zbudowany. \\
Najważniejsze jego elementy to:
\begin{itemize*}
\item Jądro - centrum obliczeniowe neuronu.
\item Dendryty - wejścia neuronu. Przesyłają do jądra sygnały poddawane późniejszej obróbce.
\item Synapsa - łączy dendryt z jądrem. W synapsie sygnał wejściowy może ulegać wstępnej modyfikacji, to znaczy być wzmacniany lub osłabiany.
\item Wzgórek aksonu - łączy jądro z aksonem.
\item Akson - wyjście neuronu. Zazwyczaj rozgałęzia się, przesyłając sygnał do wielu wejść kolejnych neuronów.
\end{itemize*}

\begin{figure}[h]
\begin{center}
\includegraphics[width=10cm]{resources/figures/natural_neuron.png}
\caption{Model budowy neuronu biologicznego}
\end{center}
\end{figure}
\label{NaturalNeuronRys}

\newpage
\subsection{Budowa sztucznego neuronu}
Porównując budowę sztucznego neuronu z neuronem biologicznym, można znaleźć wiele analogii. \\
Koncepcyjny model budowy sztucznego neuronu przedstawiono na rysunku \ref{SztucznyNeuronRys}

\vspace{1cm}
\begin{figure}[h]
\begin{center}
\includegraphics[width=15cm]{resources/figures/artificial_neuron.png}
\caption{Model budowy sztucznego neuronu}
\end{center}
\end{figure}
\label{SztucznyNeuronRys}

\subsection{Jak działa sztuczny neuron?}
Na wejścia neuronu podawane są sygnały wejściowe. Każdy sygnał wejściowy
jest przemnażany przez odpowiadającą mu wagę.
Przemnożone sygnały wejściowe są następnie ze sobą sumowane.
Sumowanie następuje w bloku sumującym.
Uzyskaną wartość nazywamy potencjałem membranowym.
Potencjał membranowy jest przekazywany do funkcji aktywacji, która na jego
podstawie oblicza wartość podawaną na wyjście neuronu.

Zachowanie neuronu jest silnie uzależnione od rodzaju wykorzystywanej funkcji
aktywacji.

\subsection{Funkcje aktywacji}
\begin{large}
!!! TODO !!!
\end{large}

\subsection{Warstwy w sieciach neuronowych}
Sieci neuronowe posiadają zazwyczaj wiele warstw. \\
Neurony należące do tej samej warstwy nie są zazwyczaj ze sobą połączone.
Warstwy sieci neuronowej dzieli się na trzy rodzaje:
\begin{itemize*}
\item warstwa wejściowa - składa się z neuronów pobierających zestaw danych
wejściowych
\item warstwa wyjściowa - składa się z neuronów generujących wynik obliczeń
sieci neuronowej
\item warstwy ukryte - znajdują się pomiędzy warstwą wejściową a warstwą wyjściową
\end{itemize*}

\subsection{Klasyfikacja sieci neuronowych}
Sieci neuronowe można klasyfikować na wiele sposobów, w zależności od przyjętego kryterium.
\begin{enumerate*}
\item Podział sieci ze względu na sprzężenia występujące pomiędzy neuronami:
\begin{itemize*}
\item Jednokierunkowe (feedforward) - sieci, w których nie występują żadne
sprzężenia zwrotne. Dane zawsze przepływają tylko w jedną stronę (od
wejścia do wyjścia).
\item Rekurencyjne (feedback) – sieci o połączeniach dwukierunkowych.
W takich sieciach dopuszczone jest połączenie zwrotne, czyli sygnał z
wyjścia może być przenoszony na wejścia neuronów z poprzednich warstw.
\item Sieci komórkowe - sprzężenia wzajemne między elementami
przetwarzającymi dotyczą jedynie najbliższego sąsiedztwa.
\end{itemize*}
\item Podział sieci ze względu na typ uczenia się:
\begin{itemize*}
\item Sieci z uczeniem nadzorowanym - algorytm uczący adaptuje sieć stosownie do wymuszeń zewnętrznych, starając się możliwie wiernie zrealizować zadaną funkcję
\item Sieci z uczeniem nienadzorowanym - sieć nie jest ograniczona żadnym zewnętrznym kryterium (funkcją błędu), zadaniem sieci jest wykrycie prawidłowości w
zbiorze danych podawanym na jej wejście
\end{itemize*}
\end{enumerate*}

\newpage
\subsection{Cechy sieci neuronowych}
Jak wykazano w pracy \cite{dudek:wyklad:sieciAproksymacja}, najważniejszymi cechami sieci neuronowych są:
\begin{enumerate*}
\item Własność uniwersalnego aproksymatora (sieci potrafią aproksymować dowolną funkcję, z dowolnie małym błędem)
\item Pozyskiwanie wiedzy z danych
\item Duża odporność na zakłócenia danych
\item Równoległa architektura (równoległe przetwarzanie danych)
\item Adaptacyjność względem otrzymywanych danych
\item Zdolność do samoorganizacji, czyli samodzielnego dostrajania swoich
parametrów, w celu lepszego dostosowania się do wykonywanych zadań
\end{enumerate*}

\subsection{Zalety i wady sieci neuronowych}

\subsection{Porównanie sieci neuronowych z tradycyjnymi modelami obliczeniowymi}
Sieci neuronowe w sposób zasadniczy różnią się od tradycyjnych modeli obliczeniowych.
 
W tradycyjnym modelu obliczeniowym, wszystkie kroki wykonywanego algorytmu muszą być ściśle zdefiniowane przez programistę.

Natomiast w przypadku sieci neuronowych, algorytm rozwiązywanego problemu może być dla programisty nieznany lub znany tylko na bardzo ogólnym poziomie.
Sieć neuronowa jest w stanie samodzielnie odnaleźć optymalny algorytm rozwiązania.

\newpage
\section{Metody uczenia sieci neuronowych}
\subsection{Uczenie nadzorowane}
\subsection{Uczenie nienadzorowane}

\section{Algorytmy uczące}
\subsection{Tradycyjnie wykorzystywane algorytmy}
\subsection{Algorytmy ewolucyjne}

\section{Proces tworzenia sieci neuronowej}
\subsection{Przygotowanie modelu sieci}
\subsection{Przygotowanie danych treningowych}
\subsection{Trening}