\chapter{Wstęp do sztucznych sieci neuronowych}
\section{Pojęcia podstawowe}
\subsection{Sztuczne sieci neuronowe}
\subsubsection{Czym są sztuczne sieci neuronowe?}
Sztuczna sieć neuronowa jest modelem obliczeniowym, którego filozofia pracy oparta jest na funkcjonowaniu biologicznego układu nerwowego.
Sztuczna sieć neuronowa składa się z dużej liczby wzajemnie ze sobą powiązanych elementów, zwanych neuronami.

Sieci neuronowe są wykorzystywane w informatyce do rozwiązywania problemów, z którymi nie radzą sobie tradycyjne modele obliczeniowe.

Najważniejszą cechą sieci neuronowych jest ich zdolność uczenia się, w oparciu o dostarczany im zbiór danych treningowych.

\subsubsection{Sieci neuronowe kontra tradycyjne modele obliczeniowe}
Sieci neuronowe w sposób zasadniczy różnią się od tradycyjnych modeli obliczeniowych.
 
W tradycyjnym modelu obliczeniowym, wszystkie kroki wykonywanego algorytmu muszą być ściśle zdefiniowane przez programistę.

Natomiast w przypadku sieci neuronowych, algorytm rozwiązywanego problemu może być dla programisty nieznany lub znany tylko na bardzo ogólnym poziomie.
Sieć neuronowa jest w stanie samodzielnie odnaleźć optymalny algorytm rozwiązania.

\subsubsection{Czym jest sztuczny neuron?}
Sztuczny neuron jest matematyczną funkcją, posiadającą wiele wejść i tylko jedno wyjście. Danymi na wejściu i wyjściu neuronu są najczęściej liczby rzeczywiste, będące  w zadanym zakresie.

Przyjęło się, aby każde wejście neuronu miało przypisaną wagę, czyli wartość liczbową określającą jak ważne jest dane wejście dla neuronu.
Bardzo często wartości wag mieszczą się w przedziale [-1; 1].

Sztuczny neuron posiada 2 tryby pracy - tryb nauki oraz tryb odtwarzania. \newline
W trybie nauki neuron może być uczony, aby na odpowiedni zestaw danych
wejściowych generował na wyjściu określoną wartość liczbową. \newline
W trybie odtwarzania, neuron klasyfikuje zestawy danych wejściowych, generując
odpowiadające im wyjścia (zgodnie z tym, czego nauczył się w trybie uczenia).

\subsubsection{Budowa neuronu biologicznego}
Ponieważ budowa sztucznego neuronu opiera się na budowie naturalnej komórki nerwowej, dlatego ważne aby zrozumieć jak neuron biologiczny jest zbudowany. \\
Najważniejsze jego elementy to:
\begin{itemize*}
\item Jądro - centrum obliczeniowe neuronu.
\item Dendryty - wejścia neuronu. Przesyłają do jądra sygnały poddawane późniejszej obróbce.
\item Synapsa - łączy dendryt z jądrem. W synapsie sygnał wejściowy może ulegać wstępnej modyfikacji, to znaczy być wzmacniany lub osłabiany.
\item Wzgórek aksonu - łączy jądro z aksonem.
\item Akson - wyjście neuronu. Zazwyczaj rozgałęzia się, przesyłając sygnał do wielu wejść kolejnych neuronów.
\end{itemize*}

\begin{figure}[h]
\begin{center}
\includegraphics[width=10cm]{resources/figures/natural_neuron.png}
\caption{Model budowy neuronu biologicznego}
\end{center}
\end{figure}

\newpage
\subsubsection{Budowa sztucznego neuronu}
Porównując budowę sztucznego neuronu z neuronem biologicznym, można znaleźć wiele analogii. \\
Koncepcyjny model budowy sztucznego neuronu przedstawiono na rysunku poniżej.

\vspace{1cm}
\begin{figure}[h]
\begin{center}
\includegraphics[width=15cm]{resources/figures/artificial_neuron.png}
\caption{Model budowy sztucznego neuronu}
\end{center}
\end{figure}
\newpage

\section{Metody uczenia sztucznych sieci neuronowych}