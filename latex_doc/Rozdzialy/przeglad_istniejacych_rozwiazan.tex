\chapter{Przegląd i analiza istniejących rozwiązań}
\section{
	Samuel Artz - "Deep Learning Cars"
	\cite{artz:deepLearningCars:blog}
	\cite{artz:deepLearningCars:github}
}

\begin{figure}[h]
\begin{center}
\includegraphics[width=16cm]{resources/figures/samuelArtzDeepCars.png}
\caption{Stopklatka z pracy systemu}
\end{center}
\end{figure}
\label{SamuelArtzDeepLearningCars}

\subsection{Opis implementacji}
Na początku każdej generacji jest tworzonych 20 samochodów (osobników). \\
Każdy samochód posiada własną sieć neuronową oraz zestaw 5 czujników, odmierzających odległość z przodu samochodu.

Sygnały z czujników wysyłane są do neuronów warstwy wejściowej.
Czujniki te służą do wykrywania przeszkód w otoczeniu samochodu.
Każdy czujnik jest skierowany w inną stronę.
Czujniki tworzą "pole widzenia" samochodu o łącznym kącie 90 stopni.

Zadaniem każdego samochodu jest bezkolizyjny przejazd po wyznaczonym torze.
Jedyne operacje jakie może wykonać samochód, to skręt w lewo lub prawo.
Prędkość wszystkich samochódów jest stała i ustalana przed rozpoczęciem generacji.
W chwili kolizji z przeszkodą, samochód umiera.
Należy dodać, iż nie występują kolizje pomiędzy samochodami.

Kiedy wszystkie samochody umrą, bieżąca generacja się kończy.
Osobniki do kolejnej generacji są tworzone przy pomocy algorytmu ewolucyjnego.

Do reprodukcji wybierane są 2 osobniki z najwyższym przystosowaniem.

Przystosowania jest obliczane na podstawie odległości przebytej przez samochód.
Im większa przejechana odległość, tym wyższa wartość przystosowania samochodu.

Geny 2 najlepiej przystosowanych osobników są krzyżowane i mutowane, tworząc 20 nowych osobników (potomków).

Genami samochodu są wagi w jego sieci neuronowej.
Zastosowane sieci neuronowe to sieci jednokierunkowe (tzw. feedforward),
z 5 pięcioma neuronami w warstwie wejściowej, 2 neuronami w warstwie wyjściowej i dwiema warstwami ukrytymi (odpowiednio 4 i 3 neurony).
Uczenie sieci (dostrajanie wartości wag) odbywa się za pomocą tzw. Algorytmu Genetycznego.

\subsection{Reprezentacja graficzna}
Dwuwymiarowa symulacja środowiska (toru) została wykonana w aplikacji Unity3D.
Krzyżyki wyświetlane na ekranie są reprezentacją czujników samochodów.
Kamera podąża za samochodem z najlepszym przystosowaniem.
Dwa samochody z najlepszym przystosowaniem są wyróżniane kolorem.

\subsection{Dostępność kodu źródłowego}
Cały kod źródłowy projektu został udostępniony przez autora. \\
Link do repozytorium znajduje się w pozycji literaturowej:  \cite{artz:deepLearningCars:github}.
\section{kwea123 - Autocar}
\section{Microsoft - AirSim}