\chapter{Przegląd i analiza istniejących rozwiązań}

Przedstawione poniżej przykłady prezentują bardzo zróżnicowane podejścia do rozwiązywanego problemu, jak również cechują się różnym stopniem zaawansowania technicznego (zarówno w warstwie logiki, jak i reprezentacji graficznej).
Cechą wspólną, łączącą wszystkie przykłady, jest tematyka tworzenia
samochodów autonomicznych (wirtualnych lub rzeczywistych).

\section{
	Samuel Artz - "Deep Learning Cars"
	\cite{artz:deepLearningCars:blog}
	\cite{artz:deepLearningCars:github}
}

\begin{figure}[h]
\begin{center}
\includegraphics[width=16cm]{resources/figures/samuelArtzDeepCars.png}
\caption{Stopklatka z pracy systemu}
\end{center}
\end{figure}
\label{SamuelArtzDeepLearningCars}

\subsection{Opis implementacji}
Na początku każdej generacji jest tworzonych 20 samochodów (osobników). \\
Każdy samochód posiada własną sieć neuronową oraz zestaw 5 czujników, odmierzających odległość z przodu samochodu.

Sygnały z czujników wysyłane są do neuronów warstwy wejściowej.
Czujniki te służą do wykrywania przeszkód w otoczeniu samochodu.
Każdy czujnik jest skierowany w inną stronę.
Czujniki tworzą "pole widzenia" samochodu o łącznym kącie 90 stopni.

Zadaniem każdego samochodu jest bezkolizyjny przejazd po wyznaczonym torze.
Jedyne operacje jakie może wykonać samochód, to skręt w lewo lub prawo.
Prędkość wszystkich samochódów jest stała i ustalana przed rozpoczęciem generacji.
W chwili kolizji z przeszkodą, samochód umiera.
Należy dodać, iż nie występują kolizje pomiędzy samochodami.

Kiedy wszystkie samochody umrą, bieżąca generacja się kończy.
Osobniki do kolejnej generacji są tworzone przy pomocy algorytmu ewolucyjnego.

Do reprodukcji wybierane są 2 osobniki z najwyższym przystosowaniem.

Przystosowania jest obliczane na podstawie odległości przebytej przez samochód.
Im większa przejechana odległość, tym wyższa wartość przystosowania samochodu.

Geny 2 najlepiej przystosowanych osobników są krzyżowane i mutowane, tworząc 20 nowych osobników (potomków).

Genami samochodu są wagi w jego sieci neuronowej.
Zastosowane sieci neuronowe to sieci jednokierunkowe (tzw. feedforward),
z 5 pięcioma neuronami w warstwie wejściowej, 2 neuronami w warstwie wyjściowej i dwiema warstwami ukrytymi (odpowiednio 4 i 3 neurony).
Uczenie sieci (dostrajanie wartości wag) odbywa się za pomocą tzw. Algorytmu Genetycznego.

\subsection{Reprezentacja graficzna}
Dwuwymiarowa symulacja środowiska (toru) została wykonana w aplikacji Unity3D.
Krzyżyki wyświetlane na ekranie są reprezentacją czujników samochodów.
Kamera podąża za samochodem z najlepszym przystosowaniem.
Dwa samochody z najlepszym przystosowaniem są wyróżniane kolorem.

\subsection{Dostępność kodu źródłowego}
Cały kod źródłowy projektu został udostępniony przez autora. \\
Link do repozytorium znajduje się w pozycji literaturowej:  \cite{artz:deepLearningCars:github}.

\newpage
\section{
	kwea123 - "Autocar"
	\cite{kwea:autocar}
}

Projekt wykonany w technologii Unity3D oraz Python.
Autor wytrenował sieci neuronowe, stosując podejście uczenia ze wzmocnieniem.
Autor wykorzystuje bibliotekę OpenCV do wykrywania krawędzi toru i spłaszcza obraz na wejściu, łącząc go z prędkością samochodu.
Sieć była trenowana, wykorzystując algorytm PPO (Proximal Policy Optimization).

Samochodem może sterować człowiek lub komputer (zwany agentem).
Zadaniem agenta jest dojazd do wyznaczonego celu, lub przejechanie bezkolizyjne całego toru.
Jazda do przodu jest wynagradzana, podczas gdy cofanie się jest karane.
Agent jest wyposażony w kamerę umieszczoną z przodu samochodu, która dostarcza mu informacji o pokonywanym torze.

Agenci są skojarzeni z obiektem Main, który odpowiada za wszystkie interakcje pomiędzy środowiskiem Unity, a skryptami napisanymi w Pythonie.
 
Projekt zawiera trzy środowiska treningowe:
\begin{itemize*}
\item autocar \\
\begin{figure}[H]
\begin{center}
\includegraphics[width=16cm]{resources/figures/autocar.png}
\caption{autocar}
\end{center}
\end{figure}
\label{Autocar}
Tor szczątkowy, przeznaczony do eksperymentów ze środowiskiem. \\
Sieć wytrenowana po 200 iteracjach.

\newpage
\item autocar2 \\
\begin{figure}[H]
\begin{center}
\includegraphics[width=16cm]{resources/figures/autocar2.png}
\caption{autocar2}
\end{center}
\end{figure}
\label{Autocar2}
Kompletny tor. \\
Sieć wytrenowana po kolejnych 200 iteracjach, wykorzystując wstępnie wytrenowaną sieć z poprzedniego środowiska treningowego.

\item autocar\_human\_detection \\
\begin{figure}[H]
\begin{center}
\includegraphics[width=15cm]{resources/figures/autocar_human.png}
\caption{autocar\_human\_detection}
\end{center}
\end{figure}
\label{Autocar_human}
Tor identyczny jak w środowisku autocar2, jednakże uzupełniony o pojawiającą się w losowych miejscach postać przechodzącego pieszego.\\
Zadaniem agenta jest unikanie kolizji z pieszym. \\
Aby to osiągnąć, obliczana jest odległość od pieszego oraz kierunek jego przemieszczenia.

Autor wykorzystuje tutaj uczenie nadzorowane z dużym zbiorem danych.
Wykrywanie pieszego jest wykonywane za pomocą frameworka 
Tensorflow Object Detection API.

Wytrenowane modele, odpowiedzialne za prognozę odległości i kierunku przemieszczania się pieszego, bazują na sieciach 
MLP (Multilayer Perceptron - Wielowarstowy Perceptron) oraz 
CNN (Convolutional Neural Network - Konwolucyjna Sieć Neuronowa).
\end{itemize*}

\newpage
\section{Microsoft - AirSim}