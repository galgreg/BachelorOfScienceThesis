\chapter*{Wstęp}
\addcontentsline{toc}{chapter}{Wstęp}
Przedmiotem pracy jest wykorzystanie technik uczenia maszynowego do rozwiązywania problemu sterowania samochodem autonomicznym. Jest to bardzo skomplikowane zagadnienie, dlatego większość uwagi została poświęcona dobremu zrozumieniu podstawowych jego aspektów.

\vspace{1.5cm}
\section*{Cel i zakres pracy}
Celem niniejszej pracy jest opracowanie prostego systemu uczącego sieci neuronowe w oparciu o symulacje przeprowadzane w wymodelowanym środowisku.   Problemem rozwiązywanym przez sieć neuronową jest nawigowanie samochodem po torze wyścigowym. Zakres pracy obejmuje następujące zagadnienia:
\begin{enumerate*}
\item Przegląd literatury na temat samochodów autonomicznych, wyzwań stojących przed ich twórcami oraz przykładów istniejących rozwiązań z tego zakresu.
\item Przegląd literatury na temat sieci neuronowych i algorytmów ewolucyjnych.
\item Wybór algorytmów ewolucyjnych wykorzystywanych podczas uczenia sieci.
\item Zaprojektowanie aplikacji oraz wybór odpowiednich narzędzi, które ułatwią jej implementację.
\item Implementacja aplikacji uczącej sieci neuronowe w oparciu o sygnały dostarczane ze środowiska symulacji.
\item Przeprowadzenie eksperymentów obliczeniowych poprzez wykorzystanie utworzonej aplikacji oraz analiza wyników uzyskanych z tych eksperymentów.
\end{enumerate*}

\newpage
\section*{Struktura pracy}
Praca składa się z pięciu numerowanych rozdziałów. Każdy rozdział dotyczy konkretnego aspektu omawianego tematu, a kolejność rozdziałów realizuje zasadę ,,\textit{od ogółu do szczegółu}''. Praca rozpoczyna się od rozdziałów omawiających ogólne zagadnienia teoretyczne, niezbędne do zrozumienia dalszych rozdziałów pracy. Kończy się natomiast rozdziałem będącym opisem bardzo konkretnych i praktycznych aspektów tematu. \\
Oto lista rozdziałów zawartych w tej pracy:
\begin{enumerate*}
\item \textbf{Samochody autonomiczne} \\
Rozdział opisujący problem samochodów autonomicznych. Celem tego rozdziału jest wyrobienie u czytelnika intuicji na temat tego, czym są samochody autonomiczne i jakie wyzwania stoją przed twórcami takich pojazdów. Podczas opisywania tego problemu zostało poruszonych wiele kwestii, których świadomość jest niezwykle ważna do odpowiedniego zrozumienia całej sprawy.
\item \textbf{Wstęp do uczenia maszynowego} \\
Rozdział zawiera najważniejsze zagadnienia teoretyczne z zakresu sztucznych sieci neuronowych oraz algorytmów ewolucyjnych. Opisane zagadnienia stanowią podstawę teoretyczną, której znajomość jest niezbędna do zrozumienia dalszych rozdziałów pracy.
\item \textbf{Projekt systemu i opis narzędzi} \\
Rozdział składa się z dwóch części. W pierwszej części zawarte są główne założenia projektowe, jakie zostały przyjęte przed rozpoczęciem prac nad aplikacją. Natomiast druga część rozdziału to opis technologii, jakie zostały wykorzystane podczas implementowania aplikacji.
\item \textbf{Opis implementacji} \\
Rozdział opisuje implementację aplikacji wykonanej na potrzeby niniejszej pracy. W rozdziale zawarte są stosunkowo szczegółowe informacje na temat poszczególnych komponentów wchodzących w skład aplikacji.
\item \textbf{Eksperymenty obliczeniowe} \\
W tym rozdziale przedstawiona jest metodyka przeprowadzania eksperymentów oraz analiza uzyskanych tą drogą wyników. Wyniki eksperymentów obliczeniowych pozwalają na wyciągnięcie pewnych wartościowych wniosków, które również zostały opisane w tym rozdziale.
\end{enumerate*}
