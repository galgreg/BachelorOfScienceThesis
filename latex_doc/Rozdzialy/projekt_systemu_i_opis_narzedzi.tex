\chapter{Projekt systemu i opis narzędzi}
W niniejszym rozdziale zostaną przedstawione najważniejsze założenia i decyzje projektowe, jakie zostały przyjęte przed rozpoczęciem prac nad aplikacją. Ponadto zamieszczam opis narzędzi, wykorzystanych podczas procesu implementacji systemu.

\section{Projekt systemu}
Wytwarzanie oprogramowania nie należy do zadań trywialnych. Dotyczy to zwłaszcza prac nad skomplikowanymi systemami informatycznymi. Podstawą sukcesu jest jasne zdefiniowanie założeń projektowych oraz odpowiednie zaplanowanie architektury systemu. Ten etap pracy najlepiej wykonać w początkowych fazach projektu, gdy podejmowanie kluczowych decyzji projektowych nie jest jeszcze obarczone wysokim kosztem.

Podczas fazy projektowania aplikacji postanowiłem rozpatrzeć trzy zagadnienia, które moim zdaniem są najistotniejsze dla dalszego zrozumienia tworzonego systemu. Są to:
\begin{enumerate*}
\item Definicja \textbf{rozwiązywanego problemu},
\item Lista \textbf{celów i założeń aplikacji},
\item \textbf{Architektura systemu}, rozważana na najwyższym poziomie abstrakcji.
\end{enumerate*}

Każde z tych zagadnień zostało omówione poniżej.

\subsection{Definicja rozwiązywanego problemu}
Problemem rozwiązywanym przez aplikację jest trening agentów sterujących samochodami wymodelowanymi w środowisku symulacji. Samochody mają pokonywać wyznaczone tory wyścigowe bez powodowania kolizji ze ścianami. Agenci odbierają informacje o swoim bieżącym położeniu dzięki sensorom odległości, umocowanym na każdym samochodzie. W odpowiedzi na te informacje agenci wysyłają do środowiska symulacji żądania wykonania określonych akcji np. zmiany kąta kierownicy w samochodzie kontrolowanym przez agenta. Oprócz zjawisk opisanych powyżej, środowisko symulacji zlicza wartości przystosowania agentów i zwraca listę tych wartości (po jednej dla każdego agenta) po każdym zakończonym epizodzie symulacji. Wartości przystosowania są bardzo ważne, gdyż bez nich byłoby niemożliwe przeprowadzenie procesu uczenia agentów.

Agentami są jednokierunkowe sieci neuronowe \cite{multilayerNN:article} o prostej topologii (domyślnie składają się z trzech warstw o wymiarach 3-5-2). Parametry sieci neuronowych (wagi i biasy) są dostrajane przy użyciu dwóch algorytmów ewolucyjnych: \textbf{Ewolucji Różnicowej} (patrz sekcja \ref{DeOverview}) oraz \textbf{PSO} (patrz sekcja \ref{PsoOverview}).

\subsection{Cele i założenia}
Każda aplikacja powinna realizować jakieś cele i założenia. Oto lista celów i założeń, jakie przyjąłem dla swojej aplikacji:
\begin{enumerate*}
\item Aplikacja powinna być prosta w obsłudze;
\item Aplikacja powinna działać na systemach Linux oraz Windows;
\item Aplikacja powinna zapisywać wytrenowane modele do pliku o ustalonym formacie. Plik ten powinien być później możliwy do odczytu przez aplikację;
\item Dane konfiguracyjne nie powinny być na sztywno zaszyte w kodzie, lecz dostarczane z zewnętrznego pliku o ustalonym formacie;
\item Aplikacja powinna dokumentować swój przebieg w plikach dziennika. Pliki dziennika powinny zawierać wszelkie istotne informacje, które mogą być pomocne przy późniejszym debugowaniu;
\item Eksperymenty obliczeniowe powinny się wykonywać przy minimalnym wysiłku użytkownika. Efektem działania tych eksperymentów powinny być wykresy oraz inne dane, które będzie można zaprezentować przy omawianiu wyników eksperymentów obliczeniowych.
\end{enumerate*}

\subsection{Architektura systemu}
\begin{figure}[H]
\centering
\includegraphics[width=15cm]{resources/figures/software_architecture.png}
\caption{Wizualizacja architektury systemu}
\label{SoftwareArchitecture}
\end{figure}
Architektura systemu jest stosunkowo prosta. Składa się z dwóch zasadniczych komponentów: \textbf{środowiska symulacji} oraz \textbf{zestawu skryptów języka Python}. Komponenty komunikują się ze sobą poprzez odpowiedni mechanizm komunikacji. Rysunek \ref{SoftwareArchitecture} przedstawia wizualizację omawianej architektury. Więcej informacji na temat poszczególnych komponentów aplikacji można odnaleźć w rozdziale \ref{ImplementationOverview}-tym. Rozdział ten jest poświęcony opisowi implementacji systemu.
\section{Zastosowane technologie}
W niniejszej sekcji opisuję technologie z których korzystałem podczas implementacji mojej aplikacji.
\subsection{Unity}
Wieloplatformowy silnik gier 2D lub 3D \cite{unity:opis}. Pozwala również na tworzenie innych materiałów interaktywnych - wizualizacje, animacje itp. Silnik został napisany w językach C/C++ (platforma uruchomieniowa) oraz C\# (Unity API). Skrypty dla silnika pisze się w języku C\#. Gry tworzone na silniku Unity mogą obsługiwać wiele platform sprzętowych i systemów operacyjnych. 
Platformy wspierane przez wersję 2019.1 \cite{unity:buildOptions}:
\begin{enumerate*}
\item Komputery osobiste (PC):
\begin{enumerate*}
\item Wspierane systemy operacyjne:
\begin{itemize*}
\item Windows,
\item Linux,
\item Mac OS X;
\end{itemize*}
\item Wspierane architektury sprzętowe:
\begin{itemize*}
\item x86 - procesor 32-bitowy,
\item x64 - procesor 64-bitowy,
\item Universal - wszystkie procesory,
\item x86 + x86\_64 (Universal) - wszystkie procesory (Linux);
\end{itemize*}
\end{enumerate*}
\item iOS;
\item Android;
\item WebGL;
\item Samsung TV;
\item Xiaomi;
\item i wiele innych.
\end{enumerate*}

Z silnikiem Unity są również kompatybilne hełmy rzeczywistości wirtualnej, takie jak Oculus Rift oraz Gear VR.

Do wersji 4.6 silnik był udostępniany na licencji płatnej lub darmowej zawierającej ograniczoną funkcjonalność, ale wraz z premierą Unity 5 prawie wszystkie funkcje silnika udostępniono w wersji darmowej dla twórców nieprzekraczających 100 tysięcy dolarów dochodów rocznie.

Unity oferuje również tzw. Asset Store, który umożliwia skorzystanie z płatnych lub darmowych komponentów takich jak tekstury lub skrypty. Silnik Unity został przeze mnie wykorzystany do stworzenia środowiska symulacji, niezbędnego w procesie treningu inteligentnych agentów.

\subsection{Języki programowania}
Język programowania jest jednym z najważniejszych narzędzi w rękach każdego programisty. Pozwala on na tworzenie oprogramowania w formie zrozumiałej dla człowieka i wykonywalnej dla maszyny.
Wykonanie kodu źródłowego jest możliwe po przetłumaczeniu go do postaci kodu maszynowego, czyli sekwencji rozkazów zrozumiałych dla procesora. Tłumaczeniem kodu źródłowego na język maszyny zajmują się specjalne programy, zwane \textit{kompilatorami} lub  \textit{interpreterami}. \\
Poniżej znajduje się krótki opis języków, które wykorzystałem podczas tworzenia aplikacji.

\subsubsection{C\#}
Język programowania stworzony przez firmę Microsoft \cite{csharp:wikipedia}. Najważniejsze cechy:
\begin{itemize*}
\item silne, statyczne typowanie;
\item obiektowość z hierarchią o jednym typie nadrzędnym - klasa System.Object;
\item mechanizm automatycznego odśmiecania pamięci - tzw. garbage collector;
\item kod źródłowy kompilowany do kodu pośredniego (Common Intermediate Language);
\item wiele platform uruchomieniowych - .NET Framework, .NET Core, Mono i in.
\end{itemize*}
Język C\# jest przeze mnie wykorzystywany do pisania skryptów dla silnika Unity.

\subsubsection{Python}
Język programowania ogólnego przeznaczenia, o rozbudowanym pakiecie bibliotek standardowych 
\cite{python:wikipedia}.
Najważniejsze cechy:
\begin{itemize*}
\item projekt Open Source;
\item język interpretowany;
\item dynamiczne typowanie - weryfikacja typu następuje w czasie wykonania programu. Twórcy języka kierowali się zasadą ,,duck typing'' (kacze typowanie) - ,,jeśli obiekt zachowuje się jak kaczka, to jest kaczką'';
\item silne typowanie - dla każdego typu dozwolone są tylko te operacje, które zostały dla niego zdefiniowane;
\item garbage collector zarządza pamięcią;
\item wsparcie dla wielu paradygmatów. W Pythonie można programować obiektowo, strukturalnie lub funkcyjnie;
\item brak wsparcia dla mechanizmu enkapsulacji;
\item prosta, czytelna składnia - bloki tworzone przez wcięcia;
\item łatwy do nauczenia się;
\item istnieje wiele implementacji - CPython, Jython, IronPython i in. Standardową implementacją jest CPython (implementacja w języku C).
\end{itemize*}
W projekcie stosuję język Python w wersji 3.6.8. Do zarządzania zależnościami pakietów, stosuję narzędzie Conda.

\subsection{Frameworki}
Framework jest spójnym zestawem modułów i bibliotek programistycznych, tworzących szkielet do budowy aplikacji danego typu. Programowanie przy użyciu frameworka polega na rozbudowywaniu tego szkieletu o dodatkowe komponenty, które są wymagane dla danego projektu. Korzystanie z frameworków ułatwia życie programistom, gdyż autorzy kodu nie muszą tracić czasu na implementację nudnych i powtarzalnych elementów aplikacji, które są identyczne dla wszystkich aplikacji tego typu. Zamiast tego mogą się skupić na implementacji funkcjonalności realizowanych przez ich oprogramowanie.
\subsubsection{Unity ML-Agents}
\label{UnityMlaDescription}
Otwartoźródłowa wtyczka do Unity, ułatwiająca tworzenie środowisk treningowych dla inteligentnych agentów \cite{unitymla:overview}. Agenci mogą być uczeni rozmaitymi technikami, m. in.:
\begin{itemize*}
\item uczenie ze wzmocnieniem (Reinforcement Learning);
\item uczenie przez naśladowanie (Imitation Learning);
\item neuroewolucja.
\end{itemize*}
Framework udostępnia Pythonowe API, pozwalające na sterowanie środowiskiem ze skryptu napisanego w Pythonie. Komunikacja pomiędzy zewnętrznym procesem Pythona, a środowiskiem Unity odbywa się poprzez mechanizm gniazd (z ang. \textit{sockets}).

Unity ML-Agents składa się z trzech zasadniczych komponentów:
\begin{enumerate*}
\item Środowisko Uczenia (Learning Environment) -- symulacja środowiska treningowego.
\item Python API -- zawiera zbiór klas i metod języka Python, pozwalających na tworzenie skryptów kontrolujących Środowisko Uczenia. Wśród kodu wchodzącego w skład API są przykładowe algorytmy uczenia maszynowego, dostarczone przez twórców frameworka. Skrypty wykorzystujące Python API nie są częścią Unity. Istnieją jako samodzielne, zewnętrzne procesy.
\item External Communicator -- zapewnia komunikację pomiędzy Środowiskiem Uczenia i zewnętrznym procesem Pythona.
\end{enumerate*}
Na rysunku \ref{UnityMlaBasicModel} został zobrazowany diagram opisanych komponentów.
Z diagramu wynika, że External Communicator wchodzi w skład Środowiska Uczenia.

\begin{figure}[h]
\begin{center}
\includegraphics[width=15cm]{resources/figures/unitymla_basic_architecture.png}
\caption{Uproszczony model frameworka Unity ML-Agents}
\label{UnityMlaBasicModel}
\end{center}
\end{figure}

\vspace{1cm}
Trening zachowań agentów w symulowanych środowiskach jest możliwy dzięki zdefiniowaniu trzech podstawowych pojęć:
\begin{enumerate*}
\item Obserwacje (Observations) -- zbiór informacji rejestrowanych przez agenta w każdym kroku symulacji. 
Istnieją dwa typy obserwacji:
\begin{itemize*}
\item Obserwacja Wektorowa (Vector Observation) -- wektor liczb zmiennoprzecinkowych.
\item Obserwacje Wizualne (Visual Observations) -- obrazy z kamer i/lub dane z renderowanych tekstur
\end{itemize*}
\item Akcje (Actions) -- instrukcje wydawane przez kod sterujący agentem. Akcje są podejmowane na podstawie obserwacji otrzymanych od agenta. Akcje mają najczęściej postać wektora liczb zmiennoprzecinkowych.
\item Sygnały nagród (Reward Signals) -- wartości liczbowe, uzyskiwane co pewien czas ze Środowiska Uczenia. Sygnały nagród są miarą wyznaczającą stopień poprawności wykonania zadań przez agenta.
\end{enumerate*}

Uproszczony opis pętli symulacji:
\begin{enumerate*}
\item Następuje krok symulacji.
\item Agent zbiera obserwacje i wysyła ,,na zewnątrz''.
\item Kod sterujący agentem wyznacza akcję na podstawie bieżących obserwacji.
\item Agent wykonuje akcję.
\item Opcjonalnie: Środowisko emituje sygnał nagrody.
\item Powrót do punktu 1.
\end{enumerate*}
Celem treningu jest zazwyczaj zmaksymalizowanie sumy zdobytych nagród.

Środowisko uczenia zawiera trzy dodatkowe komponenty, ułatwiające organizację procesu uczenia:
\begin{enumerate*}
\item Klasa Agent (agenci) -- instancje klas dziedziczących po klasie Agent są przypinane do instancji klasy GameObject. Dlatego każdy obiekt sceny Unity może być agentem. Agenci są odpowiedzialni za generowanie obserwacji, wykonywanie akcji oraz emitowanie sygnałów nagród. Każdy agent jest połączony z dokładnie jednym mózgiem, czyli obiektem klasy dziedziczącej po klasie Brain
\item Klasa Brain (mózg) -- klasa interfejsowa, odpowiedzialna za logikę podejmowania decyzji przez agentów. Każdy agent musi być przypisany do dokładnie jednego mózgu, natomiast jeden mózg może zarządzać wieloma agentami. 
Mózg podejmuje decyzje, jakie akcje dla poszczególnych obserwacji powinien wykonać agent. Ściślej mówiąc, mózg jest komponentem, który otrzymuje obserwacje od agenta i wysyła mu akcje do wykonania. \\
Framework Unity ML-Agents udostępnia 3 klasy dziedziczące po klasie Brain:
\begin{itemize*}
\item LearningBrain -- przeznaczony do uruchamiania wytrenowanych modeli oraz trenowania nowych.
\item HeuristicBrain -- przeznaczony do bezpośredniego zapisu w kodzie strategii zachowania agentów.
\item PlayerBrain -- mapuje przyciski klawiatury na konkretne akcje. Może być wykorzystywany do ręcznego testowania stworzonego Środowiska Uczenia.
\end{itemize*}
\item Klasa Academy (akademia) -- klasa interfejsowa, odpowiedzialna za organizację procesu zbierania obserwacji i podejmowania decyzji. Dla każdej sceny Unity może istnieć jeden i tylko jeden obiekt klasy dziedziczącej po klasie Academy. Akademia ma kilka istotnych parametrów, które mogą być ustawiane w oknie inspektora. \\
Komponent zewnętrznego komunikatora (External Communicator), który jest odpowiedzialny za komunikację pomiędzy Środowiskiem Uczenia a zewnętrznym procesem Pythona, znajduje się wewnątrz obiektu akademii.
Jest więc częścią klasy Academy.
\end{enumerate*}

Omówione do tej pory komponenty (Learning Environment, External Communicator, Agent, Brain, Academy oraz Python API) tworzą drzewo zależności. Liczba mózgów i agentów może być w takim drzewie dowolna, o ile będą zachowane zasady opisane powyżej (każdy mózg ma co najmniej jednego agenta, każdy agent jest przypisany do dokładnie jednego mózgu). \\ Przykłady takich drzew:
\begin{itemize*}
\item Drzewo z jednym mózgiem i dwoma agentami:
\begin{figure}[h]
\begin{center}
\includegraphics[width=15cm]{resources/figures/unitymla_learning_env_example.png}
\label{UnityMlaSimpleModel}
\end{center}
\end{figure}

\item Drzewo z wieloma mózgami i wieloma agentami:
\begin{figure}[h]
\begin{center}
\includegraphics[width=15cm]{resources/figures/unitymla_learning_env_example_2.png}
\label{UnityMlaComplexModel}
\end{center}
\end{figure}

\end{itemize*}
\newpage
Dodatkowe funkcjonalności oferowane przez framework Unity ML-Agents:
\begin{enumerate*}
\item Wsparcie dla Dockera \cite{unitymla:docker}
\item Wsparcie dla treningu w chmurze obliczeniowej:
\begin{itemize*}
\item AWS,
\item Microsoft Azure.
\end{itemize*}
\end{enumerate*}

Więcej informacji na temat omawianej wersji frameworka można znaleźć w oficjalnej dokumentacji, dostępnej pod adresem:
\begin{center}
\url{https://github.com/Unity-Technologies/ml-agents/tree/0.9.1/docs}
\end{center}

\subsubsection{PyTorch}
Otwartoźródłowy framework uczenia maszynowego przeznaczony dla języków Python i C++.
Najważniejsze cechy i funkcjonalności frameworka \cite{pytorch:features}\cite{pytorch:vs:tensorflow}:
\begin{enumerate*}
\item Prostata oraz zgodność z filozofią Zen Python \cite{python:zen}
\item Łatwy dostęp do zawartości zmiennych
\item Integracja z biblioteką NumPy - dwie proste funkcje tensor.numpy oraz tensor.from\_numpy pozwalają na proste i co ważne szybkie przekształcanie pomiędzy tablicą numpy a tensorem (tensor jest podstawowym typem tablicowym, używanym we frameworku PyTorch). Oprócz prostego przekształcania, PyTorch przejmuje po bibliotece NumPy całą filozofię pracy - większość operacji możliwych do wykonania na tablicach numpy, można również wykonać na tensorach.
\item Dynamiczne budowanie grafu obliczeniowego \cite{pytorch:understandingGraphs}
\item Ułatwione uruchamianie na GPU
\item Łatwe operowanie na zbiorach danych dzięki interfejsom Dataset i DataLoader
\item TorchScript - wsparcie dla tworzenia skompilowanych wersji tworzonego modelu.
\item Wsparcie dla standardu ONNX (Open Neural Network Exchange)
\item Oferuje 2 frontendy:
\begin{itemize*}
\item Dla Pythona
\item Dla C++
\end{itemize*}
\item Wsparcie dla wielu platform chmurowych, w tym między innymi:
\begin{itemize*}
\item Amazon Web Services
\item Google Cloud Platform
\item Microsoft Azure
\end{itemize*}
\end{enumerate*}
Framework PyTorch składa się z wielu modułów. Najistotniejszym dla mnie modułem był moduł torch.nn, który wykorzystałem do implementacji sieci neuronowej. Więcej na ten temat można znaleźć w rozdziale poświęconym implementacji systemu.

\subsection{Biblioteki}
Biblioteka programistyczna to zbiór powiązanych tematycznie jednostek kodu (takich jak klasy czy funkcje). Użycie biblioteki to sposób na ponowne wykorzystanie tego samego kodu. Dobór odpowiednich bibliotek pozwala na znaczne przyspieszenie procesu tworzenia oprogramowania. Poniżej znajduje się krótkie zestawienie bibliotek wykorzystywanych w mojej aplikacji.

\subsubsection{ddt}
\label{DdtOpis}
Pakiet dla języka Python \cite{ddt:documentation}. Pozwala na łatwe tworzenie parametryzowanych testów. Stosowany jako rozszerzenie funkcjonalności frameworków testowych, takich jak unittest (który wchodzi w skład biblioteki standardowej Pythona). Nazwa pakietu pochodzi od metodyki DDT - Data Driven Testing, która zakłada oddzielenie logiki testu od danych na których test operuje. Takie podejście pozwala w znacznym stopniu ograniczyć duplikację kodu w klasach testowych.
\subsubsection{docopt}
\label{docoptOpis}
Pakiet pomagający utworzyć interfejs wiersza poleceń \cite{docopt:documentation}. Pakiet docopt jest oparty na konwencji, wykorzystywanej przez lata dla komunikatów pomocy i stron podręcznika man. Opis interfejsu jest sformalizowanym komunikatem pomocy.
\subsubsection{matplotlib}
\label{matplotlibDescription}
Biblioteka do tworzenia wykresów dla języka Python. Uchodzi za wszechstronne, lecz trudne w opanowaniu narzędzie.
\subsection{Conda}
System open source do zarządzania pakietami i środowiskami uruchomieniowymi dla systemów Windows, macOS oraz Linux. Posiada wsparcie dla wielu języków, m. in. Python, R, Ruby, Lua, Scala \cite{conda:documentation}.
Conda potrafi szybko instalować, uruchamiać i aktualizować pakiety wraz z ich zależnościami.
Bardzo łatwo można tworzyć nowe wirtualne środowiska dla Pythona.

\subsection{Wykorzystywane wersje oprogramowania}
\begin{enumerate*}
\item Silnik Unity - 2019.1.12f
\item C\# - Mono 6.4.0.198
\item Python - 3.6.8
\item Unity ML-Agents - 0.9.1
\item PyTorch - 1.1.0
\item ddt - 1.2.1
\item docopt - 0.6.2
\item matplotlib - 3.1.1
\item Conda - 4.7.5
\end{enumerate*}

\subsection{Stacja robocza - specyfikacja techniczna}
Poniżej zamieszczam specyfikację techniczną stacji roboczej (komputera), którą wykorzystałem do implementacji systemu oraz wszystkich eksperymentów obliczeniowych:
\begin{enumerate*}
\item Typ urządzenia - Laptop
\item Marka i model urządzenia - Dell Inspiron 7559
\item Procesor - Intel(R) Core(TM) i7-6700HQ (2.6 GHz)
\item Pamięć RAM - SODIMM DDR3 Synchronous 1600 MHz (8 GB)
\item Karty graficzne:
\begin{itemize*}
\item nVidia GeForce GTX 960M
\item Intel HD Graphics 530
\end{itemize*}
\item Dysk - PLEXTOR PX-512M7 (SSD 512 GB)
\item System operacyjny - Linux Mint 19.1 Tessa
\end{enumerate*}