\chapter*{Podsumowanie}
\addcontentsline{toc}{chapter}{Podsumowanie}

Celem pracy było opracowanie prostego systemu uczącego sieci neuronowe, bazującego na symulacjach przeprowadzanych w wymodelowanym środowisku. Zadaniem sieci neuronowych było sterowanie samochodem, poruszającym się po wirtualnym środowisku symulacyjnym. Sieci neuronowe były uczone za pomocą algorytmów PSO i Ewolucji Różnicowej.

Cel pracy został zrealizowany w całości. Stworzona aplikacja umożliwia przeprowadzanie treningu sieci neuronowych. Ewolucja Różnicowa (patrz sekcja \ref{DeOverview}) oraz algorytm PSO (patrz sekcja \ref{PsoOverview}) były z sukcesami zastosowane do uczenia sieci. Wytrenowane sieci neuronowe potrafią pokonywać wymodelowane tory wyścigowe (patrz sekcja \ref{UnityScenes}), nawet te o dużym stopniu skomplikowania. Więcej informacji o wynikach uczenia sieci można odnaleźć w rozdziale \ref{ExperimentsChapter}.

\vspace{2cm}
\section*{Perspektywy dalszych badań w dziedzinie}
Działania przeprowadzone w ramach tworzenia niniejszej pracy stanowią zaledwie wstęp do badania omawianego zagadnienia. Uczenie maszynowe oraz tworzenie autonomicznych samochodów to dwa bardzo szerokie tematy, którym można poświęcić wiele lat badań.

Jednym z możliwych kierunków dalszego rozwoju byłoby przeprowadzenie eksperymentów na pojeździe poruszającym się po rzeczywistym środowisku. Takim pojazdem mógłby być model samochodu, wykonany w pewnej skali. Innym pomysłem wartym rozważenia jest poszerzenie percepcji samochodu poprzez dołączenie mechanizmu wizji komputerowej, o której opowiada artykuł \cite{computerVision:overview}. Wykorzystanie tej technologii mogłoby w znaczący sposób zwiększyć możliwości tworzonego pojazdu.

\newpage
\section*{Opinie i przemyślenia}
Tematyka poruszana w tej pracy jest bardzo ważna i istotna, nie tylko dla osób zainteresowanych nowinkami technicznymi. Uczenie maszynowe oraz szerzej pojęta sztuczna inteligencja zmienią sposób, w jaki będzie funkcjonować społeczeństwo przyszłości. Będą miały istotny wpływ na życie każdego człowieka. 

W ciągu najbliższych kilkunastu lat czeka nas wielka rewolucja przemysłowa, w wyniku której zniknie wiele wykonywanych obecnie zawodów \cite{czwartaRewolucja:artykul}. W wyniku tego, sporo ludzi będzie zmuszonych do zmiany pracy.

Istotnym tematem jest także kwestia odpowiedzialnego wykorzystania zdobyczy technologicznych. Wiele osób, uznawanych za wybitne w świecie nauki, ostrzegają przed skutkami nieetycznego wykorzystania sztucznej inteligencji. Do takich osób należy m.in. Elon Musk, który wielokrotnie wypowiadał się na ten temat \cite{elonMusk:sztucznaInteligencja}. Bardzo ważne, aby wszystkie osoby zajmujące się tą tematyką miały świadomość odpowiedzialności, jaka na nich spoczywa. Dotyczy to zwłaszcza polityków, uczestniczących w procesie tworzenia dokumentów legislacyjnych. Odpowiednie regulacje prawne z zakresu sztucznej inteligencji są niezbędne. Tylko dzięki nim jest szansa, że zwykły obywatel nie będzie czuł zagrożenia w sytuacji, w której nowoczesne technologie coraz bardziej wkraczają w jego życie codzienne.

Musimy zapewnić, aby sztuczna inteligencja pozostała narzędziem, które jest w naszej kontroli i które nie jest wykorzystywane do celów nieetycznych. Nie możemy dać się zaślepić możliwościom, jakie oferuje nam ta technologia. Nie możemy bagatelizować potencjalnych zagrożeń. Tylko dzięki wypracowaniu odpowiednich zabezpieczeń jest szansa, że sztuczna inteligencja nigdy nie odwróci się przeciwko ludzkości.

\addcontentsline{toc}{chapter}{Spis rysunków} 
\listoffigures